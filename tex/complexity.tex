\chapter{Problems complexity and approximability}%
\label{ch:complexity}

We will now prove the inapproximability of the \acrshort{ECP} and
\acrshort{D-ECP} within some nontrivial factor.

\section{Hardness of \acrshort{ECP}}%
\label{sub:ecp-hardness}

\begin{theorem}
	\label{th:approximability}
	\acrfull{ECP} has no $n^{1-\epsilon} $-approximation algorithm for
	any $\epsilon$ unless $\mathcal{P} = \mathcal{NP}  $
\end{theorem}

\begin{proof}
	We show this by presenting a direct reduction from \textsc{Maximum
		Independent Set} (MIS), which is known having the mentioned hardness
	factor (\autoref{tab:inapproximability-examples}).

	\bigskip
	Let $G_{1}  = (V_{1} ,E_{1} )$ be an undirected and unweighted graph and
	$\lambda > \frac{\alpha }{1 - \alpha }$, $\lambda \in \mathbb{N} $ and $n_{1} \coloneqq |V_{1}| $.
	We construct the \emph{Interaction Graph} ${G}_{2}  = (V_{2} , E^{+}_{2} , E
			^{-}_{2} ) $ as follows

	\begin{itemize}
		\item for each vertex $v_{i}  \in V_{1} $ we add a vertex in $G_{2} $
		\item for each edge $e_{ij}  \in
			      E_{1} $ we add $\lambda n_{1} $ negative edges between $v_{i} $ and $v_{j} $
		\item we add a vertex $v_r$ and a positive edge between it and any other
		      vertex that we already inserted in $G_2$
		\item we add a vertex $v_x$ and $\lambda n_{1} $ negative edges between $v_x$
		      and $v_{r} $
	\end{itemize}

	Furthermore, all the edges in $G_{2} $ are associated to the same content
	$C$ and the same thread $T \in \mathcal{T}_{C}  $.
	An illustration of the conversion can be found in \autoref{fig:construction}.

	\begin{figure}[b]
		\begin{center}
			\begin{subfigure}{0.4\textwidth}
				\centering
				\tikzfig{tex/tikz/approximability1}
				\vspace{10pt}
				\caption{$G_{1}$, undirected graph}
				\label{fig:g1_example}
			\end{subfigure}
			\begin{subfigure}{0.4\textwidth}
				\centering
				\tikzfig{tex/tikz/approximability2}
				\caption{$G_{2}$, directed signed graph, for $\lambda = 1$}
				\label{fig:g2_example}
			\end{subfigure}
		\end{center}
		\caption[Example reduction from MIS to \acrshort{ECP}]{Example construction of the interaction graph $G_{2} $ from
			$G_{1} $, for $\alpha = \frac{1}{3} $}
		\label{fig:construction}
	\end{figure}

	\begin{claim}
		\label{th:claim-controversial}
		Content $C$ is \emph{controversial}.
	\end{claim}
	\begin{proof}
		Let $m_{2}^{-} $ and $m_{2}^{+} $ be the number of negative and
		positive edges in $G_2$, respectively.

		By construction $m_{2}^{+} = n_{1} $ and $m_{2}^{-} \geq \lambda n_{1}
		$ (the negative edges between $v_r$ and $v_x$). Also, for $a, b, c \in \mathbb{R}^{+}$ it holds that $\frac{a +
				b}{a + b + c} \geq \frac{a}{a + c} $. Consequently

		\begin{align}
			\eta(C) = \frac{m_{2}^{-} }{m_{2}^{-} +
				m_{2}^{+} } \geq \frac{\lambda n_{1}}{\lambda n_{1}
				+ n_{1} } = \frac{\lambda }{\lambda + 1} > \alpha
		\end{align}
	\end{proof}

	So content $C$ is \emph{controversial}. This reduces the \acrlong{ECP} on $G_2$ to the maximization of

	\begin{equation}
		\label{eq:score}
		\xi(U) = \sum^{}_{T \in \mathcal{S}_C(U) } (| T[U]^{+} | - | T[U]^{-} |)
	\end{equation}

	\begin{claim}
		\label{th:opt-equality}
		\begin{equation}
			OPT(ECP) = OPT(MIS)
		\end{equation}
	\end{claim}

	\begin{proof}
		Let $I \subseteq V_{1} $ be an independent set of $G_1$ of size $|I| >
			1$. Consider the associated solution in $G_2$ in which $U = I
			\cup \{v_{r} \}$. By construction, it will contain only $|I|$ positive
		edges, so $T[U] \in \mathcal{S}_C(U)$ and also

		\begin{equation}
			OPT(ECP) \geq \xi(U) = |T^{+}[U]| = |I| \implies OPT(ECP) \geq OPT(MIS)
		\end{equation}

		Now let $S \subseteq V_2$ be a solution of the \acrshort{ECP} on
		$G_2$, and suppose $\xi(S) > 0$. It is easy to see that $v_{r} \in S$
		and that $v_{x} \not\in S $. Let $J \coloneqq S \setminus \{v_r\}$.

		Suppose that $2$ vertices $v_{i} $, $v_{j} \in J$ are linked in
		$G_1$. By construction there are at least $\lambda n_1$ negative edges
		in $T[S]$, thus

		\begin{equation}
			\eta(T[S]) \geq \frac{\lambda n_1}{\lambda n_1 + |S-1|} \geq \frac{\lambda n_1}{\lambda n_1 + n_1} = \frac{\lambda
			}{\lambda + 1} > \alpha
		\end{equation}

		This means that $T[S]$ is \emph{controversial} $\implies \xi(S) = 0
			\implies contradiction$ and, consequently, $J$
		contains vertices which are independent in $G_1$. Therefore, $T[S]$ contains
		only positive edges; more specifically

		\begin{equation}
			\xi(S) = |T^{+}[S]| = |S| - 1 = |S \setminus \{v_r\}| = |J|
		\end{equation}

		Thus

		\begin{equation}
			OPT(MIS) \geq |J| \implies OPT(MIS) \geq OPT(ECP)
		\end{equation}

		So the optimal value of the constructed instance of \acrshort{ECP}
		exactly equals that of the \textsc{Maximum Independent Set} instance, so it
		has a hardness factor at least as large as that of MIS.
	\end{proof}

	This concludes the proof of \autoref{th:approximability}.
\end{proof}

\section{Hardness of \acrshort{D-ECP}}%
\label{sub:d-ecp-hardness}

\begin{theorem}
	\label{th:approximability-densest}
	\acrfull{D-ECP} has no $n^{1-\epsilon} $-approximation algorithm for
	any $\epsilon$ unless $\mathcal{P} = \mathcal{NP}  $
\end{theorem}

\begin{proof}
	We again show this results by presenting a direct reduction from \textsc{Maximum
		Independent Set}.

	\bigskip
	Let $G_{1}  = (V_{1} ,E_{1} )$ be an undirected and unweighted graph and
	$\lambda > \frac{\alpha }{1 - \alpha }$, $\lambda \in \mathbb{N} $ and $n_{1} \coloneqq |V_{1}| $.
	We construct the \emph{interaction} graph ${G}_{2}  = (V_{2} , E^{+}_{2} , E
			^{-}_{2} ) $ as follows

	\begin{itemize}
		\item for each vertex $v_{i}  \in V_{1} $ we add a vertex in $G_{2} $
		\item for each edge $e_{ij}  \in
			      E_{1} $ we add $\lambda (n_{1}+1)^{2}  $ negative edges between $v_{i} $ and $v_{j} $
		\item for each edge $e_{ij} \in V_1 \times V_1, e_{ij} \not\in
			      E_{1} $ we add $2$ positive edges between $v_{i} $ and $v_{j} $
		\item we add a vertex $v_r$ and $2$ positive edges between it and any other
		      vertex that we already inserted in $G_2$
		\item we add a vertex $v_x$ and $\lambda n_{1}^{2}  $ negative edges between $v_x$
		      and $v_{r} $
	\end{itemize}

	Furthermore, all the edges in $G_{2} $ are associated to the same content
	$C$ and the same thread $T \in \mathcal{T}_{C}  $.
	An illustration of the conversion can be found in
	\autoref{fig:construction-densest}.

	\begin{figure}
		\begin{center}
			\begin{subfigure}[b]{0.4\textwidth}
				\centering
				\tikzfig{tex/tikz/approximability1-densest}
				\vspace{30pt}
				\caption{$G_{1}$, undirected graph}
				\label{fig:g1_example}
			\end{subfigure}
			\begin{subfigure}[b]{0.4\textwidth}
				\centering
				\tikzfig{tex/tikz/approximability2-densest}
				\caption{$G_{2}$, directed signed graph}
				\label{fig:g2_example}
			\end{subfigure}
		\end{center}
		\caption[Example reduction from MIS to \acrshort{D-ECP}]{Example construction of the interaction graph $G_{2} $ from
			$G_{1} $}
		\label{fig:construction-densest}
	\end{figure}

	\begin{claim}
		\label{th:claim-controversial-densest}
		Content $C$ is \emph{controversial}.
	\end{claim}
	\begin{proof}
		% Let $m_{2}^{-} $ and $m_{2}^{+} $ be the number of negative and
		% positive edges in $G_2$, respectively.
		%
		By construction $m_{2}^{+} \leq n_1 (n_1 -1 ) \leq n_{1}^{2}  $ and $m_{2}^{-} \geq
			\lambda n_{1}^{2} $.
		Thus

		\begin{align}
			\eta(C) = \frac{m_{2}^{-} }{m_{2}^{-} +
				m_{2}^{+} } \geq \frac{\lambda n_{1} ^{2} }{\lambda n_{1}^{2}
				+ n_{1}^{2}  } = \frac{\lambda }{\lambda + 1}
			> \alpha
		\end{align}
	\end{proof}

	So content $C$ is \emph{controversial}. This reduces the \acrlong{D-ECP} on $G_2$ to the maximization of

	\begin{equation}
		\label{eq:score-densest}
		\psi(U) = \sum^{}_{T \in \mathcal{S}_C(U) } \frac{| T^{+}[U] | - | T^{-}[U] |}{|U|}
	\end{equation}

	\begin{claim}
		\label{th:opt-equality-densest}
		\begin{equation}
			OPT(D-ECP) = OPT(MIS)
		\end{equation}
	\end{claim}

	\begin{proof}
		Let $I \subseteq V_{1} $ be an independent set of $G_1$ of size $n_{I}
			\coloneqq |I| > 1$. Consider the associated solution in $G_2$ in
		which $U = I \cup \{v_{r} \}$.

		By construction, it will contain only positive edges, more specifically
		\begin{itemize}
			\item $2 \cdot n_{I} $ edges between $v_{r} $ and $v_{i} \in I$
			\item $n_{I}(n_{I}  -1)$
			      edges between vertices $v_{i} \in I$

		\end{itemize}
		thus $T[U] \in \mathcal{S}_C(U)$ and also

		\begin{equation}
			\label{eq:score-densest-mip}
			\psi(U) = \frac{|T^{+}[U]| - |T^{-}[U]|}{|U|}  = \frac{2n_{I}  +
				\cdot n_{I}(n_{I}  -1) }{n_{I} + 1} = \frac{n_{I}^{2} +
				n_{I}}{n_{I} + 1} = n_{I}
		\end{equation}

		Consequently,
		\begin{equation}
			OPT(D-ECP) \geq \psi(U) = |I| \implies OPT(D-ECP) \geq OPT(MIS)
		\end{equation}

		Now let $S \subseteq V_2$ be a solution of the \acrshort{D-ECP} on $G_2$, and suppose $\psi(S) > 0$. It is easy to see that
		$v_{r} \in S$ and that $v_{x} \not\in S $. Let $J \coloneqq S \setminus
			\{v_r\}$.

		Suppose that $2$ vertices $v_{i} $, $v_{j} \in J$ are linked in $G_1$.
		By construction there are at least $\lambda (n_1 + 1)^{2} $ negative edges in
		$T[S]$, thus

		\begin{equation}
			\eta(T[S]) \geq \frac{\lambda (n_1+1)^2}{\lambda (n_1+1)^2 +
				n_j(n_j+1)} \geq
			\frac{\lambda (n_1+1)^{2} }{\lambda (n_1+1)^2 + (n_1+1)^2} = \frac{\lambda }{\lambda +
				1} > \alpha
		\end{equation}

		where $n_{j} \coloneqq |J|$. This means that $T[S]$ is \emph{controversial} $\implies \psi(S) = 0
			\implies contradiction$. Consequently, $J$
		contains vertices which are independent in $G_1$. Therefore, $T[S]$ contains
		only positive edges. As shown previously in
		\autoref{eq:score-densest-mip}

		\begin{equation}
			\psi(S) = \frac{|T^{+}[S]|}{|S|} = |J|
		\end{equation}

		Thus,
		\begin{equation}
			OPT(MIS) \geq |J| \implies OPT(MIS) \geq OPT(D-ECP)
		\end{equation}

		So the optimal value of the constructed instance of \acrshort{D-ECP}
		exactly equals that of the \textsc{Maximum Independent Set} instance, so it
		has a hardness factor at least as large as that of MIS.
	\end{proof}

	This concludes the proof of \autoref{th:approximability-densest}.
\end{proof}

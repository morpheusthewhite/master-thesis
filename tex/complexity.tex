\chapter{Problem Complexity and Approximability}%
\label{ch:complexity}

We will now prove the inapproximability of the \acrshort{ECP} and
\acrshort{D-ECP} within some nontrivial factor.

\section{Hardness of \acrshort{ECP}}%
\label{sub:ecp-hardness}

\begin{theorem}
	\label{th:approximability}
	The \acrfull{ECP} has no $n^{1-\epsilon} $-approximation algorithm for
	any $\epsilon > 0$ unless $\mathcal{P} = \mathcal{NP}  $.
\end{theorem}

\begin{proof}
	We show this by presenting a direct reduction from \textsc{Maximum
		Independent Set} (MIS), which is known having the mentioned hardness
	factor (\autoref{tab:inapproximability-examples}).

	\bigskip
	Let $G_{1}  = (V_{1} ,E_{1} )$ be an undirected and unweighted graph for
	which we want to solve MIS.

	We show how to construct an interaction graph \mbox{${G}_{2}$} as instance for \acrshort{ECP} with parameter
	$\alpha $. Let $\lambda > \frac{\alpha }{1 - \alpha }$, $\lambda \in \mathbb{N} $ and $n_{1} \coloneqq |V_{1}| $.
	$G_2$ is constructed as follows:

	\begin{itemize}
		\item for each vertex $v_{i}  \in V_{1} $ we add a vertex in $G_{2} $,
		\item for each edge $e_{ij}  \in
			      E_{1} $ we add $\lambda n_{1} $ negative edges between $v_{i}
		      $ and $v_{j} $,
		\item we add a vertex $v_r$ and a positive edge between $v_r$ and any other
		      vertex $v_i \in V_2$ that we already inserted in $G_2$,
		\item we add a vertex $v_x$ and $\lambda n_{1} $ negative edges between $v_x$
		      and $v_{r} $.
	\end{itemize}

	Furthermore, all the edges in $G_{2} $ are associated to the same content
	$C$ and the same thread $T \in \mathcal{T}_{C}  $.
	Thus, our \acrshort{ECP} instance only contains a single thread and a
	single content.  An illustration of the reduction can be found in \autoref{fig:construction}.

	\begin{figure}
		\begin{center}
			\begin{subfigure}{0.4\textwidth}
				\centering
				\tikzfig{tex/tikz/approximability1}
				\vspace{10pt}
				\caption{$G_{1}$, undirected graph}
				\label{fig:g1_example}
			\end{subfigure}
			\begin{subfigure}{0.4\textwidth}
				\centering
				\tikzfig{tex/tikz/approximability2}
				\caption{$G_{2}$, directed signed graph, for $\lambda = 1$}
				\label{fig:g2_example}
			\end{subfigure}
		\end{center}
		\caption[Example reduction from MIS to \acrshort{ECP}]{Example construction of the interaction graph $G_{2} $ from
			$G_{1} $, for $\alpha = \frac{1}{3} $.}
		\label{fig:construction}
	\end{figure}

	\begin{claim}
		\label{th:claim-controversial}
		Content $C$ is \emph{controversial}, i.e.\ $\eta(C) > \alpha $.
	\end{claim}
	\begin{proof}
		Let $m_{2}^{-} $ and $m_{2}^{+} $ be the number of negative and
		positive edges in $G_2$, respectively.

		By construction in $G_2$ there is exactly one
		positive edge between $v_r$ and each vertex $v_i$ from $ G_1$, i.e.\
		$m_{2}^{+} = n_{1} $. Also,
		$m_{2}^{-} \geq \lambda n_{1} $, since $G_2$ contains at least the $\lambda
			n_1$ negative edges between $v_r$ and $v_x$. Consequently, given that for $a, b, c \in \mathbb{R}^{+}$ it holds that $\frac{a +
				b}{a + b + c} \geq \frac{a}{a + c} $, we have

		\begin{align}
			\eta(C) = \frac{m_{2}^{-} }{m_{2}^{-} +
				m_{2}^{+} } \geq \frac{\lambda n_{1}}{\lambda n_{1}
				+ n_{1} } = \frac{\lambda }{\lambda + 1} > \alpha.
		\end{align}
	\end{proof}

	Thus, the content $C$ is \emph{controversial}. Since our instance only contains
	a single content, this reduces the \acrshort{ECP} on $G_2$ to the maximization of

	\begin{equation}
		\label{eq:score}
		\xi(U) = \sum^{}_{T \in \mathcal{S}_C(U) } (| T[U]^{+} | - | T[U]^{-} |).
	\end{equation}

	\bigskip

	\begin{claim}
		\label{th:opt-equality}
		Let $\operatorname{OPT}(\operatorname{ECP})$ and
		$\operatorname{OPT}(\operatorname{MIS})$ be the maximum Echo Chamber score on
		$G_2$ and the size of the MIS on $G_1$, respectively.
		We have that
		\begin{equation}
			\operatorname{OPT}(\operatorname{ECP}) = \operatorname{OPT}(\operatorname{MIS})
		\end{equation}
	\end{claim}

	\begin{proof}
		Let $I \subseteq V_{1} $ be an independent set of $G_1$ of size $|I| >
			1$. Consider the associated solution in $G_2$ in which $U = I \cup
			\{v_{r} \}$. By construction, $T[U]$ only contains $|I|$ positive
		edges, so $T[U] \in \mathcal{S}_C(U)$ and also

		\begin{equation}
			\operatorname{OPT}(\operatorname{ECP}) \geq \xi(U) = |T^{+}[U]| =
			|I| \implies \operatorname{OPT}(\operatorname{ECP}) \geq
			\operatorname{OPT}(\operatorname{MIS}).
		\end{equation}

		Now let $S \subseteq V_2$ be a solution of the \acrshort{ECP} on $G_2$,
		and suppose $\xi(S) > 0$. We will have that $v_{r} \in S$ and that
		$v_{x} \not\in S $. Let $J \coloneqq S \setminus \{v_r\}$.

		Next, we argue that $J$ is an independent set for $G_1$. We prove this by
		contradiction. Suppose that two vertices $v_{i} $, $v_{j} \in J$ are
		linked in $G_1$. By construction there are at least $\lambda n_1$
		negative edges in $T[S]$, thus

		\begin{equation}
			\eta(T[S]) \geq \frac{\lambda n_1}{\lambda n_1 + |S-1|} \geq \frac{\lambda n_1}{\lambda n_1 + n_1} = \frac{\lambda
			}{\lambda + 1} > \alpha.
		\end{equation}
		This means that $T[S]$ is \emph{controversial} and $T \not\in
			\mathcal{S}_C(S) $; therefore, the sum in \eqref{eq:score} resolves to
		zero, which is a \emph{contradiction}.

		Consequently, $J$ contains vertices which are independent in $G_1$.
		Therefore, $T[S]$ contains only positive edges; more specifically,

		\begin{equation}
			\xi(S) = |T^{+}[S]| = |S| - 1 = |S \setminus \{v_r\}| = |J|.
		\end{equation}

		Thus

		\begin{equation}
			\operatorname{OPT}(\operatorname{MIS}) \geq |J| \implies
			\operatorname{OPT}(\operatorname{MIS}) \geq
			\operatorname{OPT}(\operatorname{ECP}).
		\end{equation}

		So the optimal value of the constructed instance of \acrshort{ECP}
		exactly equals that of the \textsc{Maximum Independent Set} instance.

	\end{proof}
	So, if we were able to approximate \acrshort{ECP} within
	$n^{1-\epsilon}, \; \epsilon > 0$ we would be also able to approximate MIS within the
	same factor, which is not possible unless $\mathcal{P} = \mathcal{NP}
	$, given also that our reduction takes polynomial time.

	This means \acrshort{ECP} has a hardness factor at least as large as that of MIS.

	This concludes the proof of \autoref{th:approximability}.
\end{proof}

\section{Hardness of \acrshort{D-ECP}}%
\label{sub:d-ecp-hardness}

\begin{theorem}
	\label{th:approximability-densest}
	The \acrfull{D-ECP} has no $n^{1-\epsilon} $-approximation algorithm for
	any $\epsilon > 0$ unless $\mathcal{P} = \mathcal{NP}  $.
\end{theorem}

\begin{proof}
	We again show this result by presenting a direct reduction from \textsc{Maximum
		Independent Set}. Differently from before, we will need to create an
	instance of the \acrshort{D-ECP} with a positive clique over
	independent vertices of the original graph.
	% (in order to match the values of
	%     the two problems)

	\bigskip
	Let $G_{1}  = (V_{1} ,E_{1} )$ be an undirected and unweighted graph for
	which we want to solve MIS.

	We show how to construct an interaction graph \mbox{${G}_{2}$} as instance
	for \acrshort{D-ECP} with parameter $\alpha $. Let $\lambda > \frac{\alpha
		}{1 - \alpha }$, $\lambda \in \mathbb{N} $ and $n_{1} \coloneqq |V_{1}| $.
	$G_2$ is constructed as follows:

	\begin{itemize}
		\item for each vertex $v_{i}  \in V_{1} $ we add a vertex in $G_{2} $,
		\item for each edge $e_{ij}  \in
			      E_{1} $ we add $\lambda (n_{1}+1)^{2}  $ negative edges
		      between $v_{i} $ and $v_{j} $,
		\item for each edge $e_{ij} \in V_1 \times V_1 \setminus
			      E_{1} $ we add $2$ positive edges between $v_{i} $ and $v_{j}
		      $,
		\item we add a vertex $v_r$ and $2$ positive edges between $v_r$ and any other
		      vertex $v_i \in V_2$ that we already inserted in $G_2$,
		\item we add a vertex $v_x$ and $\lambda n_{1}^{2}  $ negative edges between $v_x$
		      and $v_{r} $.
	\end{itemize}

	Furthermore, all the edges in $G_{2} $ are associated to the same content
	$C$ and the same thread $T \in \mathcal{T}_{C}  $. Thus, our \acrshort{D-ECP} instance only contains a single thread and a
	single content.
	An illustration of the reduction can be found in
	\autoref{fig:construction-densest}.

	\begin{figure}
		\begin{center}
			\begin{subfigure}[b]{0.4\textwidth}
				\centering
				\tikzfig{tex/tikz/approximability1-densest}
				\vspace{30pt}
				\caption{$G_{1}$, undirected graph}
				\label{fig:g1_example}
			\end{subfigure}
			\begin{subfigure}[b]{0.4\textwidth}
				\centering
				\tikzfig{tex/tikz/approximability2-densest}
				\caption{$G_{2}$, directed signed graph}
				\label{fig:g2_example}
			\end{subfigure}
		\end{center}
		\caption[Example reduction from MIS to \acrshort{D-ECP}]{Example construction of the interaction graph $G_{2} $ from
			$G_{1}. $}
		\label{fig:construction-densest}
	\end{figure}

	\begin{claim}
		\label{th:claim-controversial-densest}
		The content $C$ is \emph{controversial}, i.e.\ $\eta(C) > \alpha $.
	\end{claim}
	\begin{proof}
		By construction $G_2$ will contain at most two positive edges between each
		pair of vertices from $G_1$ and $v_r$, i.e.\ $m_{2}^{+} \leq n_1 (n_1 +1 ) <
			(n_{1} + 1)^{2}  $. Also, $m_{2}^{-} \geq
			\lambda (n_{1} + 1)^{2} $ since $G_2$ contains at least the $\lambda
			n_1^2$ negative edges we added between $v_r$ and $v_x$.
		Thus, given that for $a, b, c \in \mathbb{R}^{+}$ it holds that $\frac{a +
				b}{a + b + c} \geq \frac{a}{a + c} $, we have that

		\begin{equation}
			\eta(C) = \frac{m_{2}^{-} }{m_{2}^{-} +
				m_{2}^{+} } \geq \frac{\lambda (n_{1} + 1) ^{2} }{\lambda
				(n_{1} + 1)^{2}
				+ (n_{1} + 1)^{2}  } = \frac{\lambda }{\lambda + 1}
			> \alpha.
		\end{equation}
	\end{proof}

	Thus, the content $C$ is \emph{controversial}. Since our instance contains
	a single content, this reduces the \acrshort{D-ECP} on $G_2$ to the maximization of

	\begin{equation}
		\label{eq:score-densest}
		\psi(U) = \sum^{}_{T \in \mathcal{S}_C(U) } \frac{| T^{+}[U] | - |
			T^{-}[U] |}{|U|}.
	\end{equation}

	\begin{claim}
		\label{th:opt-equality-densest}
		Let $\operatorname{OPT}(\operatorname{ECP})$ and
		$\operatorname{OPT}(\operatorname{MIS})$ be the maximum Echo Chamber score on
		$G_2$ and the size of the MIS on $G_1$, respectively.
		We have that
		\begin{equation}
			\operatorname{OPT}(\operatorname{D-ECP}) =
			\operatorname{OPT}(\operatorname{MIS}).
		\end{equation}
	\end{claim}

	\begin{proof}
		Let $I \subseteq V_{1} $ be an independent set of $G_1$ of size $n_{I}
			\coloneqq |I| > 1$ (unless $G_1$ is a clique we can always trivially find an independent set of
		size two by choosing two vertices that are not
		connected by an edge). Consider the associated solution in $G_2$ in
		which $U = I \cup \{v_{r} \}$.

		By construction, $T[U]$ only contains positive edges, more
		specifically:
		\begin{itemize}
			\item $2 \cdot n_{I} $ positive edges between $v_{r} $ and $v_{i}
				      \in I$,
			\item $n_{I}(n_{I}  -1)$
			      edges between vertices $v_{i} \in I$.

		\end{itemize}
		Thus $T[U] \in \mathcal{S}_C(U)$ and also

		\begin{equation}
			\label{eq:score-densest-mip}
			\psi(U) = \frac{|T^{+}[U]| - |T^{-}[U]|}{|U|}  = \frac{2n_{I}  +
				n_{I}(n_{I}  -1) }{n_{I} + 1} = \frac{n_{I}^{2} +
				n_{I}}{n_{I} + 1} = n_{I}.
		\end{equation}

		Consequently,
		\begin{equation}
			\operatorname{OPT}(\operatorname{D-ECP}) \geq \psi(U) = |I|
			\implies \operatorname{OPT}(\operatorname{D-ECP}) \geq
			\operatorname{OPT}(\operatorname{MIS}).
		\end{equation}

		Now let $S \subseteq V_2$ be a solution of the \acrshort{D-ECP} on
		$G_2$, and suppose $\psi(S) > 0$ (we can always choose $S = \{ v_r \}
			\cup \{ v_i\}$ with $v_i$ vertex from $G_1$, which will produce
		$\psi(S) = 1$). We will have that
		$v_{r} \in S$ and that $v_{x} \not\in S $. Let $J \coloneqq S \setminus
			\{v_r\}$ be the corresponding solution for MIS.

		Next, we argue that $S$ is an independent set for $G_1$. We prove this by
		contradiction. Suppose
		that two vertices $v_{i} $, $v_{j} \in J$ are linked in $G_1$.
		By construction there are at least $\lambda (n_1 + 1)^{2} $ negative edges in
		$T[S]$, thus

		\begin{align*}
			\eta(T[S]) & = \frac{|T^{-}[S]|}{|T^{-}[S]| + |T^{+}[S]|}      \\
			           & \geq \frac{\lambda (n_1+1)^2}{\lambda (n_1+1)^2 +
				n_j(n_j+1)}
			\\ & \geq \frac{\lambda (n_1+1)^{2} }{\lambda (n_1+1)^2 + (n_1+1)^2}
			\\ & = \frac{\lambda }{\lambda + 1}
			\\ & > \alpha
		\end{align*}
		where $n_{j} \coloneqq |J|$.


		This means that $T[S]$ is \emph{controversial} and $T \not\in
			\mathcal{S}_C(S) $; therefore, the sum in \eqref{eq:score-densest} resolves to
		zero, which is a \emph{contradiction}.

		Consequently, $J$ contains vertices which are independent in $G_1$.
		Therefore, $T[S]$ contains only positive edges. Similarly to
		\eqref{eq:score-densest-mip},

		\begin{equation}
			\psi(S) = \frac{|T^{+}[S]|}{|S|} = |J|.
		\end{equation}

		Thus,
		\begin{equation}
			\operatorname{OPT}(\operatorname{MIS}) \geq |J| \implies
			\operatorname{OPT}(\operatorname{MIS}) \geq
			\operatorname{OPT}(\operatorname{D-ECP})
		\end{equation}

	\end{proof}
	So the optimal value of the constructed instance of \acrshort{D-ECP}
	exactly equals that of the \textsc{Maximum Independent Set} instance.
	As motivated before (\autoref{sub:ecp-hardness}), it will have a hardness factor at least as large as that of MIS.

	This concludes the proof of \autoref{th:approximability-densest}.
\end{proof}

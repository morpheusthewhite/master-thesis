\chapter{Background}
\label{ch:background}

This chapter provides the background knowledge relevant for the thesis work. It
will initially discuss graph and related problems (\autoref{sec:signed_graphs_and_density}) which are significant in the
following used methodologies, as well as concepts of computational complexity
(\autoref{sec:computational_complexity_and_approximability}) and Linear
Programming (\autoref{sec:linear_and_mixed_integer_programming}).

\section{Signed graphs and density}%
\label{sec:signed_graphs_and_density}

\section{Computational complexity and approximability}%
\label{sec:computational_complexity_and_approximability}

Complexity Theory deals with the study of the intrinsic complexity of
computational tasks; more specifically it mainly aims at
determining the complexity of any given task. It also elaborates on the
relationships between the complexity of different problems, for example proving
that 2 problems are computationally equivalent\cite{9780521884730}.

\subsection{Complexity classes}%
\label{par:complexity_classes}

According to their complexity, problems can be divided in different groups
\cite{DemaineFall2014}.

\paragraph{$\mathcal{P} $}%
\label{par:p}
is the set of problems which can be solved in polynomial time in the size $n$ of
the problem, i.e. $n^{O(1)} $.

\paragraph{$\mathcal{NP} $}%
\label{par:np} is the set of problems whose solution can be verified in polynomial time in the size $n$ of the problem, i.e. $n^{O(1)} $.

According to these definition it is easy to see that $\mathcal{P} \subseteq
	\mathcal{NP} $.

\paragraph{$\mathcal{NP} $-Hard}%
\label{par:_np_hard} is the set of problems that are \emph{at least as hard} as
any other problem in $\mathcal{NP} $.

\paragraph{$\mathcal{NP} $-Complete}%
\label{par:_np_hard} is the set of problems in $\mathcal{NP} $-Hard that are
also in $\mathcal{NP} $. Intuitively these correspond the most difficult
problems to solve in $\mathcal{NP} $.

\subsection{$\mathcal{P} $ vs $\mathcal{NP}$}%
\label{sub:_p_vs_np_}


\section{Linear and Mixed Integer Programming}%
\label{sec:linear_and_mixed_integer_programming}

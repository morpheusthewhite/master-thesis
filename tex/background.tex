\chapter{Background}
\todo[inline, backgroundcolor=aqua]{Bakgrund}
\label{ch:background}
\todo[inline]{When you do your literature study, you should have a nearly complete Chapters 1 and 2.\\
	You may also find it convenient to introduce the future work section into your report early – so that you can put things that you think about but decide not to do now into this section.\\
	Note that later you can move things between this future work section and what you have done as you may change your mind about what to do now versus what to put off to future work.
}

This chapter provides basic background information about xxx. Additionally, this chapter describes xxx. The chapter also describes related work xxxx.

What does a reader (another x student -- where x is your study line) need to know to understand your report?
What have others already done? (This is the “related work”.) Explain what and
how prior work / prior research will be applied on or used in the degree
project /work (described in this thesis). Explain why and what is not used in
the degree project and give valid reasons for rejecting the work/research.

\todo[inline, backgroundcolor=aqua]{Vilken viktig litteratur och
	(forsknings-)artiklar har du studerat inom området (litteraturstudie)? }

\section{Major background area 1}\todo[inline, backgroundcolor=aqua]{Viktigt bakgrundsområde 1}
There are xxx characteristics that distinguish yyy from other information and communication technology (ICT) system, as shown in Figure~\ref{fig:lotsofstars}. Table \ref{tab:tablecaracteristics} summarizes these characteristics.


\todo[inline, backgroundcolor=aqua]{Massor av stärnor (Inspirerad av figur x.y på sidan z i [xxx])}


\begin{table}[!ht]
	\begin{center}
		\caption{xxx characteristics}
		\label{tab:tablecaracteristics}
		\begin{tabular}{l|S[table-format=4.6]} % <-- Alignments: 1st column left, 2nd middle, with vertical lines in between
			\textbf{Characteristics} & \textbf{Description} \\
			$\alpha$                 & $\beta$              \\
			\hline
			1                        & 1110.1               \\
			2                        & 10.1                 \\
			3                        & 23.113231            \\
		\end{tabular}
	\end{center}
\end{table}
\todo[inline, backgroundcolor=aqua]{Egenskaper}
\todo[inline, backgroundcolor=aqua]{ Beskrivning}

\subsection{Subarea 1.1}
Entangled states are an important part of quantum cryptography, but also relevant in other domains. This concept might be relevant for neutrinos, see for example \cite{kim_small-mass_2016}.

\subsection{Subarea 1.1.2}
Computational methods are increasingly used as a third method of carrying out
scientific investigations. For example, computational experiments were used to
find the amount of wear in a polyethylene liner of a hip prosthesis in \cite{maguire_jr_new_2014}.
…

\subsection{Subarea 1.1.2}
Using the nearest data center may improve performance, see \cite{bogdanov_nearest_2015}


\subsection{Link layer Encapsulation}
\label{sec:llencap}

See Figure~\ref{fig:ieee8023-data-packet} which uses the \textsf{bytefield}  \LaTeX\ package.


\begin{figure}[!ht]
	\centering
	\begin{bytefield}{21}
		\bitbox[]{7}{} & \bitbox[]{3}{\tiny octets:} & \bitbox[]{4}{\tiny 6} & \bitbox[]{4}{\tiny 6} & \bitbox[]{3}{\tiny 2} & \bitbox[]{5}{\tiny 46 to 1500} & \bitbox[]{3}{\tiny 0 to 46} & \bitbox[]{2}{\tiny 4}\\

		\bitbox[]{8}{\textbf{ETHERNET \\[-1ex] \tiny{data link-layer}}} & \bitbox[]{2}{} &

		\bitbox{4}{\tiny Destination Address} & \bitbox{4}{\tiny Source Address} & \bitbox{3}{\tiny Length/ Type} &
		\bitbox{5}{\tiny Data Payload} & \bitbox{3}{\tiny Padding} &
		\bitbox{2}{\tiny CRC} \\

		\bitbox[]{1}{} &\bitbox[]{3}{\tiny octets:} & \bitbox[]{4}{\tiny 7} & \bitbox[]{2}{\tiny 1} & \bitbox[]{0}{$\vdots$ \\[1ex]} & \bitbox[]{16}{} & \bitbox[]{0}{$\vdots$ \\[1ex]} & \bitbox[]{5}{} & \bitbox[]{4}{\tiny Variable}\\

		\bitbox[]{4}{\textbf{MAC \\[-1ex] \tiny{packet}}} & \colorbitbox{lightgrey}{4}{\tiny Preamble} & \colorbitbox{lightgrey}{2}{\tiny SFD} & \colorbitbox{lightgrey}{16}{\tiny MAC Client Data} & \colorbitbox{lightgrey}{3}{\tiny Padding} &
		\colorbitbox{lightgrey}{2}{\tiny CRC} & \colorbitbox{lightgrey}{4}{\tiny Extension}
	\end{bytefield}
	\label{fig:ieee8023-data-packet}
	\caption{Ethernet data link layer protocol encapsulated into a IEEE~802.3 MAC packet}
\end{figure}

\subsection{IP packet headers}
\label{sec:ipheaders}
The data link layer will receive a packet from the IP layer. The layout of
an IPv4 packet is shown in Figure~\ref{fig:ipv4-header}. This should be
contrasted with the IPv6 header shown in Figure~\ref{fig:ipv6-header}.

%
% IPv4 packet header
%
\begin{figure}[!ht]
	\centering
	\begin{bytefield}{32}
		\bitheader{0-31} \\
		\bitbox{4}{\footnotesize{Version}} & \bitbox{4}{IHL} & \bitbox{6}{\tiny{Type of Service}} & \bitbox{2}{{\scriptsize ECN}} \bitbox{16}{Total Length}\\
		\bitbox{16}{Identification} & \bitbox{3}{Flags} & \bitbox{13}{Fragment Offset}\\
		\bitbox{8}{Time to Live} & \bitbox{8}{Protocol} & \bitbox{16}{Header Checksum}\\
		\wordbox{1}{Source Address}\\
		\wordbox{1}{Destination Address}\\
		\colorbitbox{lightgrey}{24}{Options} & \colorbitbox{lightgrey}{8}{Padding}
	\end{bytefield}
	\label{fig:ipv4-header}
	\caption[IPv4 datagram header]{IPv4 datagram header. Light grey coloured fields are optional.}
\end{figure}

%
% IPv6 packet header
%
\begin{figure}[!ht]
	\centering
	\begin{bytefield}{32}
		\bitheader{0-31} \\
		\bitbox{4}{\footnotesize{Version}} & \bitbox{8}{Traffic Class} & \bitbox{20}{Flow Label}\\
		\bitbox{16}{Payload Length} & \bitbox{8}{Next Header} & \bitbox{8}{Hop Limit}\\
		\wordbox{4}{Source Address}\\
		\wordbox{4}{Destination Address}\\
	\end{bytefield}
	\label{fig:ipv6-header}
	\caption{IPv6 datagram header}
\end{figure}

...
\section{Major background area 2}\todo[inline, backgroundcolor=aqua]{Viktigt bakgrundsområde 2}
...

\section{Related work area}\todo[inline, backgroundcolor=aqua]{Relaterande arbeten}


\subsection{Major related work 1}\todo[inline, backgroundcolor=aqua]{Relaterande arbeten 1}
Carrier clouds have been suggested as a way to reduce the delay between the users and the cloud server that is providing them with content. However, there is a question of how to find the available resources in such a carrier cloud. One approach has been to disseminate resource information using an extension to OSPF-TE, see Roozbeh, Sefidcon, and Maguire \cite{roozbeh_resource_2013}.


\subsection{Major related work}\todo[inline, backgroundcolor=aqua]{Relaterande arbeten}

\subsection{Minor related work 1}\todo[inline, backgroundcolor=aqua]{Mindre relaterat arbete 1}


…
\subsection{Minor related work n}\todo[inline, backgroundcolor=aqua]{Mindre relaterat arbete n}


\section{Summary}\todo[inline, backgroundcolor=aqua]{Sammanfattning}
\todo[inline, backgroundcolor=aqua]{Det är trevligt att få detta kapitel
	avslutas med en sammanfattning. Till exempel kan du inkludera en tabell som
	sammanfattar andras idéer och fördelar och nackdelar med varje - så som
	senare kan du jämföra din lösning till var och en av dessa. Detta kommer
	också att hjälpa dig att definiera de variabler som du kommer att använda
	för din utvärdering.}

\todo[inline]{It is nice to have this chapter conclude with a summary. For
	example, you can include a table that summarizes other people's ideas and
	benefits and drawbacks with each - so as later you can compare your solution
	to each of them. This will also help you define the variables that you will
	use for your evaluation.}

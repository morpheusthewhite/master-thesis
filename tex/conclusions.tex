\chapter{Conclusions and Future work}
\label{ch:conclusionsAndFutureWork}

In this research we proposed a new method for detecting polarization and echo
chambers in social medias, the \acrshort{ECP} and \acrshort{D-ECP}. We
initially showed that these problems cannot be approximated even within a
non-trivial factor ($n^{1-\epsilon}$) and proposed methods for solving and
approximating them, focusing on the rounding algorithm. We observed that it is
able to find echo chambers in generated datasets but has limitations on
real-world data. We motivate the poor performances on social media datasets
with noise introduced by edge classification, sparseness of the data and
possible limitations of the specific analyzed datasets. Nonetheless, our formulation paves the way for a richer and more
expressive analysis of social media interactions, with the introduction
of the concepts of contents and threads.

Future works on the field should take into account these limitations which
may require enhancing the graph through additional information like the use
of a \emph{follow} graph and/or
changing the problem formulation in order to take into account the structure
of real-world data and overcome the intrinsic complexity of the problem.

Moreover, we proposed alternative formulations and approximation algorithms
whose performances and results could be analyzed in future research to get a
better grasp of the problem. Also, we leave open the matter regarding the
choice of $\alpha$ and how it affects the results. Finally, we leave as future
challenge the study of methods for approximating the \acrshort{D-ECP} as well
as a comparison with the results obtained with the \acrshort{ECP}.

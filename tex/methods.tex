\chapter{Solving the \acrshort{ECP} and the \acrshort{D-ECP}}
\label{ch:solving}

% This chapter focuses on describing the main topic of the
% research: the problems are motivated and defined in
% \autoref{sec:the_echo_chamber_problem}, and their complexity is then analyzed
% in \autoref{sec:problem_complexity_and_approximability}, while proposing some
% alternative problem definitions. Algorithms and models for solving and
% approximating the problems are then proposed in
% \autoref{sec:solving_the_problem}; the definition of a generative model
% concludes the chapter (\autoref{sec:generative_model}).

% \section{The \acrlong{ECP}}%
% \label{sec:the_echo_chamber_problem}

% This section starts by outlining the graph on which the analysis is carried
% out (\autoref{sub:interaction-graph}), the process for building it
% (\autoref{sub:collection_and_preprocessing}) and finally defines the
% \acrlong{ECP} and the \acrlong{D-ECP} (\autoref{sub:the_problem_definition} and
% \autoref{sub:the_densest_echo_chamber_problem}).

We now present some techniques for calculating exactly and approximating both
the \acrshort{ECP} and the \acrshort{D-ECP} (\autoref{sub:the_problem_definition}).

\section{Exact solutions}%
\label{sec:exact-solutions}

We start with our exact algorithms that are based on MIPs.

\subsection{A MIP model for the \acrshort{ECP}}%
\label{sub:a_mip_model_for_the_ecp}

Let $G$ be an \emph{interaction graph} for contents $\mathcal{C} $ and
threads $T \in \mathcal{T}_{C}, \; C \in \mathcal{C} $ for which we want to
solve the \acrshort{ECP}. Fix
$\alpha \in [0, 1]$. Let $\mathcal{\hat{C}} \subseteq \mathcal{C} $ be the
set of controversial contents and $E_k$ the set
of all edges of thread $T_k$ associated to a controversial content, i.e. $T_{k}
	\in \mathcal{T}_{C}, C \in \mathcal{\hat{C}}$; let also $E^{+}_k $
and $E^{-}_k $ be the set of positive and negative edges in $T_k$, respectively.

The following \acrshort{MIP} model is able to solve the \acrshort{ECP} on $G$
for values of $\alpha \leq 0.5$.

\begin{alignat}{3}
	\label{eq:ecp-exact1}
	\text{maximize}     &                                                 & \; \sum_{ T_{k} \in \mathcal{T}_{C}, \; C \in
		\mathcal{\hat{C}} } \big( \sum^{}_{ij \in E^{+}_k } x_{ij}
	^{k} - \sum_{ij \in E^{-}_k } x_{ij} ^{k} \big)                                                                                                                                               \\
	\label{eq:ecp-v1}
	\text{subject to}   & \quad                                           & x _{ij}^{k}  \leq y_i                                               & \quad \forall ij \in E_k                        \\
	\label{eq:ecp-v2}
	                    &                                                 & x _{ij}^{k}  \leq y_j                                               & \quad \forall ij \in E_k                        \\
	\label{eq:ecp-t1}
	                    &                                                 & x _{ij}^{k}  \leq z_k                                               & \quad \forall ij \in E_k                        \\
	\label{eq:ecp-e1}
	                    &                                                 & x _{ij} ^{k} \geq - 2 + y_i + y_j + z_k                             & \quad \forall ij \in E_k                        \\
	\label{eq:ecp-alpha-constraint1}
	                    &                                                 & \sum^{}_{ij \in E_k^{-} } x_{ij}^{k}  - \alpha \sum^{}_{ij \in E_k}
	x_{ij} ^{k}  \leq 0 & \quad \forall T_{k} \in \mathcal{T} _{C}, C \in
	\hat{\mathcal{C}}                                                                                                                                                                             \\
	\label{eq:ecp-vertex-def1}
	                    &                                                 & y _{i} \in  \{0, 1\}                                                & \quad \forall i \in V                           \\
	\label{eq:ecp-edge-def1}
	                    &                                                 & 0 \leq x _{ij} ^{k}  \leq 1                                         & \quad \forall ij \in E_k                        \\
	\label{eq:ecp-thread-def1}
	                    &                                                 & 0 \leq z _{k} \leq 1                                                & \quad \forall T_{k} \in \mathcal{T} _{C}, C \in
	\hat{\mathcal{C}}
\end{alignat}

We will now show that the \acrshort{ECP} can be solved through
MIP~\eqref{eq:ecp-exact1}-\eqref{eq:ecp-thread-def1}.

\begin{theorem}
	\label{th:ecp-mip}
	Let $G = (V, E^{+}, E^{-})$ be an \emph{Interaction Graph} and $\alpha \in
		[0, 0.5]$. Then

	\begin{equation}
		\max_{U \subseteq V} \xi(U) = \operatorname{OPT}(\operatorname{MIP}),
	\end{equation}
	where $\operatorname{OPT}(\operatorname{MIP})$ denotes the optimal solution to
	MIP~\eqref{eq:ecp-exact1}-\eqref{eq:ecp-thread-def1}.
\end{theorem}

\begin{proof}
	We will show the equality by first proving that $RHS \geq LHS$ and then
	that $LHS \geq RHS$.

	\begin{claim}
		\label{th:claim-v-b-xi}
		For any $U \subseteq V$, the
		MIP~\eqref{eq:ecp-exact1}-\eqref{eq:ecp-thread-def1} achieves value at
		least $\xi(U)$.
	\end{claim}

	\begin{proof}
		Let $E_k[U]$ the set of edges induced by $U$ in thread $T_k$. We
		construct a MIP solution as follows:

		\begin{alignat}{2}
			\label{eq:ecpp-y}
			y_i        & = \begin{cases}
				1, & \text{if } v_{i} \in U, \\
				0, & \text{otherwise,}
			\end{cases} & \forall v_i \in V                                \\
			\label{eq:ecpp-z}
			z_k        & = \begin{cases}
				1, & \text{if } T_{k} \in \mathcal{S}(U), \; C \in
				\mathcal{\hat{C}} ,                                \\
				0, & \text{otherwise,}
			\end{cases} & \forall T_k \in \mathcal{T}_C, C \in \mathcal{C} \\
			\label{eq:ecpp-x}
			x_{ij}^{k} & = \begin{cases}
				1, & \text{if } e_{ij} \in E_{k}[U], \; T_{k} \in \mathcal{S}(U), \; C \in
				\mathcal{\hat{C}},                                                         \\
				0, & \text{otherwise.}
			\end{cases} & \forall e_{ij}^k \in E_k
		\end{alignat}

		To satisfy \eqref{eq:ecp-v1}-\eqref{eq:ecp-e1} we need that
		\begin{equation}
			\label{eq:ecpp-x-iff}
			x_{ij}^{k} = 1 \iff y_i = 1 \land y_j = 1 \land z_k = 1.
		\end{equation}
		This is always true since we defined $x_{ij}^{k}$ to be $1$ only and if it
		is associated to an edge induced in a $T_k \in \mathcal{S}(U), \; C \in
			\mathcal{\hat{C}} $.

		Let us now consider a thread $T_k \in \mathcal{S}(U)$. Then

		\begin{equation}
			\eta(T_k[U]) \leq \alpha \implies \frac{|E^{-}_{k}[U]|}{|E_{k}[U]|} \leq
			\alpha \implies |E^{-}_{k}[U]| - \alpha |E_{k}[U]| \leq 0.
		\end{equation}
		so \eqref{eq:ecp-alpha-constraint1} is satisfied. It is easy to see that
		if $T_k \not\in \mathcal{S}(U)$ then $x_{ij}^{k} = 0$ for all $ij \in E_{k}$ and
		the constraint is also satisfied.

		Finally, any edge contributing to $\xi(U)$ will also
		equally contribute to the objective function.
	\end{proof}

	\begin{claim}
		\label{th:claim-xi-b-v}
		Given a feasible solution of
		MIP~\eqref{eq:ecp-exact1}-\eqref{eq:ecp-thread-def1} with value $v$ we can
		construct $U \; s.t. \; \xi(U) \geq v$.
	\end{claim}

	\begin{proof}
		We define $U \coloneqq \{ v_{i} \; s.t. \; y_i = 1\}$. Again, by
		\eqref{eq:ecp-v1}-\eqref{eq:ecp-e1} we have
		\eqref{eq:ecpp-x-iff}, so

		\begin{gather}
			x_{ij}^{k} = 1 \implies z_k = 1, \\
			\label{eq:ecpp-z-iff-xij}
			z_k = 1 \implies x_{i'j'}^{k} = 1 \; \forall i'j'
			\in E_k[U],
		\end{gather}
		meaning that if $z_k = 1$ then for all the edges $e_{ij} \in E_k$ induced
		by $U$ we will have $x_{ij}^{k} = 1$ (i.e.\ they will contribute to the
		objective).
		Let us now consider $T_k \; s.t. \; z_k = 1$. Because of
		\eqref{eq:ecp-alpha-constraint1} and \eqref{eq:ecpp-z-iff-xij} we have

		\begin{equation}
			\label{eq:ecpp-nc-t}
			|E^{-}_{k}[U]| - \alpha |E_{k}[U]| \leq 0 \implies
			\frac{|E^{-}_{k}[U]|}{|E_{k}[U]|} \leq \alpha.
		\end{equation}
		i.e. $\eta(T_k[U]) \leq \alpha$. So $T_k \in
			\mathcal{S}(U), \; C \in \mathcal{\hat{C}}$, i.e. $z_k = 1 \implies T_k
			\in \mathcal{S}(U), \; C \in \mathcal{\hat{C}}$, thus $T_k[U]$ contributes to
		$\xi(U)$; more specifically any edge contributing to the objective function equally
		contributes to $\xi(U)$.

		Now suppose there exists $T_k \in \mathcal{S}(U) \; s.t.
			\; z_k = 0$. Then $ x_{ij}^{k} = 0$ by \eqref{eq:ecp-t1}. Since $\alpha \leq 0.5$,

		\begin{align}
			\eta(T_k[U]) \leq \alpha & \implies \frac{|E^{-}_{k}[U]|}{|E_{k}[U]|}
			\leq 0.5                                                                                            \\
			                         & \implies |E^{-}_{k}[U]| \leq 0.5 \cdot (|E^{+}_{k}[U]| + |E^{-}_{k}[U]|) \\
			                         & \implies 0.5 \cdot |E^{+}_{k}[U]| - 0.5 \cdot |E^{-}_{k}[U]| \geq 0      \\
			\label{eq:ecpp-nc-contr}
			                         & \implies |E^{+}_{k}[U]| - |E^{-}_{k}[U]| \geq 0
		\end{align}
		Consequently $T_k$ will contribute positively to $\xi(U)$, i.e.
		the number of positive edges is greater or equal than the
		number of negative edges in the subgraph induced on $T_k$.

		More generally, due to \eqref{eq:ecp-v1}-\eqref{eq:ecp-e1} we have
		\begin{gather*}
			z_k = c > 0 \implies x_{i'j'}^{k} = c \; \forall i'j'
			\in E_k[U],
			% \\
			% x_{ij}^{k} = c \iff z_k = c \land y_i = 1 \land y_j = 1,
		\end{gather*}

		meaning that if $z_k = c$ all and only the variables $x_{ij}^{k}$
		associated to edges $e_{ij} \in E_k$ induced by $U$
		will get value $c$ (the others $x_{ij}^{k}$ will get value $0$ for
		\eqref{eq:ecp-v1}-\eqref{eq:ecp-v2}).

		Combining this result with \eqref{eq:ecp-alpha-constraint1} we get
		again \eqref{eq:ecpp-nc-t} and, consequently, \eqref{eq:ecpp-nc-contr}.
		Therefore, the contribution of $T_k$ associated to $z_k > 0$ will be

		\begin{equation}
			\sum^{}_{ij \in E^{+}_k } x_{ij}
				^{k} - \sum_{ij \in E^{-}_k } x_{ij} ^{k} = z_k (|E^{+}_{k}[U]| -
			|E^{-}_{k}[U]|) \geq 0.
		\end{equation}
		Since $z_k \in [0, 1]$ the constribution of the same thread in $\xi(U)$
		will be greater (if $z_k \in [0, 1)$) or equal (if $z_k = 1$) to the
		contribution of thread $T_k$ in the objective function.

		% can be similarly shown to be
		% \emph{non-controversial}; furthermore its positive contribution in the
		% objective function will be $\leq$ to $T_k$ contribution in $\xi(U)$, so
		% $\xi(U) \geq v$.
	\end{proof}

	This concludes the proof for \autoref{th:ecp-mip}.
\end{proof}

\clearpage

Solving the problem for $\alpha \in [0, 1]$ requires the definition of
additional variables and constraints

\begin{equation}
	\label{eq:ecp-exact2}
	\text{maximize} \; \sum_{ T_{k} \in \mathcal{T}_{C}, \; C \in
		\mathcal{\hat{C}} } \big( \sum^{}_{ij \in E^{+} (T_{k})} x_{ij}
		^{k} - \sum_{ij \in E^{-} (T_{k})} x_{ij} ^{k} \big)
\end{equation} \begin{center} subject to \end{center}
\begin{gather}
	\label{eq:ecp-v12}
	x _{ij}^{k}  \leq y_i \quad\quad \forall ij \in E_k  \\
	\label{eq:ecp-v22}
	x _{ij}^{k}  \leq y_j \quad\quad \forall ij \in E_k \\
	\label{eq:ecp-t2}
	x _{ij}^{k}  \leq z_k \quad\quad \forall ij \in E_k \\
	\label{eq:ecp-e2}
	x _{ij} ^{k} \geq - 2 + y_i + y_j + z_k \quad\quad \forall ij \in E_k \\
	\label{eq:ecp-a-alpha-constraint}
	-N_k z_k < \sum^{}_{ij \in E^{-} (T_k)} a_{ij}^{k}  - \alpha \sum^{}_{ij
		\in E(T_k)}
	a_{ij} ^{k} \leq - M_{k} (1 - z_{k})  \quad\quad \forall T_{k} \in
	\mathcal{T} _{C}, C \in \mathcal{\hat{C}} \footnotemark\\
	\label{eq:ecp-a-ij-g-i-j2}
	a_{ij}^{k} \geq -1 + y_i + y_j \quad\quad \forall ij \in E_k \\
	\label{eq:ecp-a-ij-l-i2}
	a_{ij}^{k} \leq y_i\quad\quad \forall ij \in E_k \\
	\label{eq:ecp-a-ij-l-j2}
	a_{ij}^{k} \leq y_j \quad\quad \forall ij \in E_k \\
	\label{eq:ecp-a-domain-2}
	0 \leq a_{ij}^{k} \leq 1 \quad\quad \forall ij \in E_k \\
	\label{eq:ecp-vertex-def2}
	y _{i} \in  \{0, 1\} \quad\quad \forall i \in V \\
	\label{eq:ecp-edge-def2}
	0 \leq x _{ij} ^{k}  \leq 1 \quad\quad \forall ij \in E_k \\
	\label{eq:ecp-z-domain-2}
	z _{k} \in  \{0, 1\} \quad\quad \forall T_{k} \in \mathcal{T} _{C}, C \in
	\hat{\mathcal{C}}
\end{gather}

\footnotetext{\label{fn:epsilon-constraint}A strict inequality is not allowed by the definition of
	\acrshort{MIP} or \acrshort{LP}, but this constraint can easily be transformed in valid
	and equivalent formulation by the mean of a small $\epsilon > 0$, so that
	it becomes
	\begin{equation*}
		-N_k z_k < \sum^{}_{ij \in E^{-} (T_k)} a_{ij}^{k}  - \alpha \sum^{}_{ij
			\in E(T_k)}
		a_{ij} ^{k} - \epsilon \leq M_{k} (1 - z_{k})  \quad\quad \forall T_{k} \in
		\mathcal{T} _{C}, C \in \mathcal{\hat{C}}
	\end{equation*}
	We will however use \eqref{eq:ecp-a-alpha-constraint} for simplifying notation
}

Where are $N_k$ and $M_k$ are constants of value $\alpha (|E_k^{+}| + 1)$ and $(1 -
	\alpha ) (|E^{-}_k| + 1)$, respectively.

\begin{theorem}
	\label{th:ecp-mip}
	Let $G = (V, E^{+}, E^{-})$ be an \emph{Interaction Graph} and $\alpha \in
		[0, 1]$.

	\begin{equation}
		\max_{U \subseteq V} \xi(U) = \text{OPT(MIP)}
	\end{equation}

	where $OPT(MIP)$ denotes the optimal solution to
	MIP~\eqref{eq:ecp-exact2}-\eqref{eq:ecp-z-domain-2}.
\end{theorem}
\begin{proof}
	\begin{claim}
		\label{th:claim-v-b-xi2}
		For any $U \subseteq V$, the
		MIP~\eqref{eq:ecp-exact1}-\eqref{eq:ecp-thread-def1} gets value $\geq \xi(U)$.
	\end{claim}

	It is easy to see that by choosing $x_{ij}^{k}, \; y_i, \; z_k$ as before and
	\begin{equation*}
		a_{ij}^{k} = \begin{cases}
			1 & \text{if } e_{ij} \in E_k[U] \\
			0 & \text{otherwise}
		\end{cases}
	\end{equation*}
	all the constraints of the new formulation are satisfied and
	Claim~\ref{th:claim-v-b-xi} is consequently proved for
	MIP~\eqref{eq:ecp-exact2}-\eqref{eq:ecp-z-domain-2}.

	We will instead focus of proving the analogous Claim of~\ref{th:claim-xi-b-v}.
	\begin{claim}
		\label{th:claim-xi-b-v2}
		Given a feasible solution of
		MIP~\eqref{eq:ecp-exact2}-\eqref{eq:ecp-z-domain-2} with value $v$ we can
		construct $U \; s.t. \; \xi(U) \geq v$ for any $\alpha$.
	\end{claim}

	\begin{proof}
		Let $U \coloneqq \{ v_i \; s.t. \; y_i = 1\} $. We won't prove
		some results in Claim~\ref{th:claim-xi-b-v} which still hold.

		Due to \eqref{eq:ecp-a-ij-g-i-j2}-\eqref{eq:ecp-a-ij-l-j2} we
		have that
		\begin{equation}
			a_{ij}^{k} = 1 \iff y_i = 1 \land y_j = 1
		\end{equation}
		i.e. all and only the edges induced by $U$ will have the corresponding
		$a_{ij}^{k} = 1$.

		Consider now a thread $T_k \in \mathcal{S}(U)$. By definition, we
		have that
		\begin{gather*}
			\eta(T_k) \leq \alpha \\
			\frac{|E^{-}_{k}[U]|}{|E_{k}[U]|} \leq \alpha \\
			|E^{-}_{k}[U]| - \alpha \cdot |E_{k}[U]| \leq 0
		\end{gather*}
		thus, because of \eqref{eq:ecp-a-alpha-constraint} $z_k = 1$. This
		means that the constraint resolves to
		\begin{equation*}
			-\alpha (|E^{+}_{k}[U]| + 1)< \sum^{}_{ij \in E^{-} (T_k)} a_{ij}^{k}  - \alpha \sum^{}_{ij \in E(T_k)} a_{ij} ^{k}
		\end{equation*}
		which is satisfied since
		\begin{multline}
			\sum^{}_{ij \in E^{-} (T_k)} a_{ij}^{k}  - \alpha \sum^{}_{ij \in E(T_k)}
			a_{ij} ^{k} = |E^{-}_{k}[U]| - \alpha \cdot |E_{k}[U]| = \\
			(1- \alpha)|E^{-}_{k}[U]| - \alpha \cdot |E^{+}_{k}[U]| \geq -
			\alpha \cdot |E^{+}_{k}[U]| > - N_k
		\end{multline}

		Consequently, because of \eqref{eq:ecpp-z-iff-xij}, we have that
		\begin{equation}
			T_k \in \mathcal{S}(U) \iff x_{ij}^{k} = 1 \; \forall \;i, j, \in U
		\end{equation}
		meaning that $T_k$ \emph{non-controversial} will equally contribute to
		$\xi$ and $v$.
		Now consider $T_k$ \emph{controversial}. In order for $v$ to be a
		feasible solution $z_k = 0$ (to satisfy
		\eqref{eq:ecp-a-alpha-constraint}, as the sum between the $2$
		inequalities is $> 0$), thus $x_{ij}^{k} = 0 \; \forall \;i, j, \in V$
		and the contribution of $T_k$ is again the same in $\xi$ and $v$ and,
		in general $\xi(U) = v$, proving the claim.
	\end{proof}

	Due to Claims~\ref{th:claim-v-b-xi2} and \ref{th:claim-xi-b-v2} the theorem
	is proved.
\end{proof}

\subsection{A MIP model for the \acrshort{D-ECP}}%
\label{sub:a_mip_model_for_the_d_ecp}

Similarly to the \acrshort{ECP}, here we propose a \acrshort{MIP} model for
finding a solution for the \acrshort{D-ECP}, $\alpha \in [0, 1]$.

For simplifying notation we define $E_{k} \coloneqq E(T_{k}), T_{k} \in
	\mathcal{T}_{C}, C \in \mathcal{\hat{C}}$.

\begin{equation}
	\label{eq:d-ecp-objective}
	\text{maximize} \; \sum_{ T_{k} \in \mathcal{T}_{C}, \; C \in
		\mathcal{\hat{C}} } \big( \sum^{}_{ij \in E^{+} (T_{k})} x_{ij}
		^{k} - \sum_{ij \in E^{-} (T_{k})} x_{ij} ^{k} \big)
\end{equation} \begin{center} subject to \end{center}
\begin{gather}
	\label{eq:d-ecp-a-ij-l-bi}
	a_{ij}^{k} \leq b_{i} \quad\quad \forall ij \in E_k \\
	\label{eq:d-ecp-a-ij-l-bj}
	a_{ij}^{k} \leq b_{j} \quad\quad \forall ij \in E_k \\
	\label{eq:d-ecp-a-ij-g-ijk}
	a _{ij} ^{k} \geq - 1 + b_i + b_j \quad\quad \forall ij \in E_k \\
	\label{eq:d-ecp-alpha-constraint}
	-N_{k} z_k < \sum^{}_{ij \in E^{-} (T_k)} a_{ij}^{k}  - \alpha \sum^{}_{ij \in E(T_k)}
	a_{ij} ^{k}  \leq M_k (1 - z_k) \quad\quad \forall T_{k} \in \mathcal{T} _{C}, C \in
	\hat{\mathcal{C}}\footnotemark \\
	\label{eq:d-ecp-edge-charikar1}
	x _{ij}^{k}  \leq y_i \quad\quad \forall ij \in E_{k} \\
	\label{eq:d-ecp-edge-charikar2}
	x _{ij} ^{k} \leq y_j \quad\quad \forall ij \in E_k \\
	\label{eq:d-ecp-vertex-charikar1}
	\sum^{}_{i \in V} y_i = 1 \\
	\label{eq:d-ecp-vertex-l-b}
	y_i \leq b_i \quad\quad \forall i \in V \\
	\label{eq:d-ecp-vertex-g-bi-yj}
	y_i \geq -1 + b_i + y_j \quad\quad \forall i,j \in V \\
	\label{eq:d-ecp-x-l-sum1}
	x_{ij}^{k} \geq -2 + a_{ij} ^{k} + z_k + y_i \quad \forall ij \in E_k \\
	\label{eq:d-ecp-x-l-sum2}
	x_{ij}^{k} \geq -2 + a_{ij} ^{k} + z_k + y_j \quad \forall ij \in E_k \\
	\label{eq:d-ecp-x-l-a}
	x_{ij} ^{k} \leq a_{ij} ^{k} \quad\quad \forall ij \in E_k \\
	\label{eq:d-ecp-x-l-z}
	x_{ij} ^{k} \leq z_k \quad\quad \forall T_{k} \in \mathcal{T} _{C}, C \in
	\hat{\mathcal{C}}  \\
	\label{eq:d-ecp-a-ij}
	a _{ij} ^{k}  \in \{0, 1\} \quad\quad \forall ij \in E_k\\
	\label{eq:d-ecp-b-i}
	b _{i} \in \{0, 1\} \quad\quad \forall i \in V \\
	\label{eq:d-ecp-y-i}
	y _{i} \geq 0 \quad\quad \forall i \in V \\
	\label{eq:d-ecp-x-ij}
	x _{ij} ^{k}  \geq 0 \quad\quad \forall ij \in E_k\\
	\label{eq:d-ecp-z-k}
	z _{k} \in \{0, 1\} \quad\quad \forall T_{k} \in \mathcal{T} _{C}, C \in
	\hat{\mathcal{C}}
\end{gather}

\footnotetext{See footnote of \autoref{sub:a_mip_model_for_the_ecp}}

Where are $N_k$ and $M_k$ are constants of value $\alpha (|E_k^{+}| + 1)$ and $(1 -
	\alpha ) ( |E^{-}_k| + 1)$, respectively.

\begin{theorem}
	\label{th:d-ecp-mip}
	Let $G = (V, E^{+}, E^{-})$ be an \emph{Interaction Graph} and $\alpha \in
		[0, 1]$.

	\begin{equation}
		\max_{U \subseteq V} \psi(U) = \text{OPT(MIP)}
	\end{equation}

	where $OPT(MIP)$ denotes the optimal solution to
	MIP~\eqref{eq:d-ecp-objective}-\eqref{eq:d-ecp-z-k}.
\end{theorem}

\begin{proof}
	We will similarly to \autoref{th:ecp-mip} and proof this equality by $2$
	inequalities: $RHS \geq LHS$ and then that $LHS \geq RHS$

	\begin{claim}
		For any $U \subseteq V$, the
		MIP~\eqref{eq:d-ecp-objective}-\eqref{eq:d-ecp-z-k} gets value $\geq \psi(U)$.
	\end{claim}

	\begin{proof}
		Let $c = 1 / |U|$. We construct a feasible solution for
		MIP~\eqref{eq:d-ecp-objective}-\eqref{eq:d-ecp-z-k} as follows

		\begin{gather}
			\label{eq:ecpp-y}
			y_i = \begin{cases}
				c & \text{if } v_{i} \in U \\
				0 & \text{otherwise}
			\end{cases} \quad
			b_i = \begin{cases}
				1 & \text{if } v_{i} \in U \\
				0 & \text{otherwise}
			\end{cases} \\
			\label{eq:ecpp-z}
			z_k = \begin{cases}
				1 & \text{if } T_{k} \in \mathcal{S}(U), \; C \in \mathcal{\hat{C}} \\
				0 & \text{otherwise}
			\end{cases} \\
			\label{eq:ecpp-x}
			a_{ij}^{k} = \begin{cases}
				1 & \text{if } e_{ij} \in E_{k}[U], \; T_{k} \in \mathcal{T} _C, \; C \in
				\mathcal{\hat{C}}                                                         \\
				0 & \text{otherwise}
			\end{cases} \\
			x_{ij}^{k} = \begin{cases}
				c & \text{if } e_{ij} \in E_{k}[U], \; T_{k} \in \mathcal{S}(U), \; C \in
				\mathcal{\hat{C}}                                                         \\
				0 & \text{otherwise}
			\end{cases}
		\end{gather}

		\eqref{eq:d-ecp-a-ij-l-bi}-\eqref{eq:d-ecp-a-ij-g-ijk} are easily
		satisfied since $a_{ij}^{k} = 1 \iff b_i = 1 \land b_j = 1$, meaning
		that it is induced $\iff v_i, v_j \in U$; the same idea applies to
		\eqref{eq:d-ecp-edge-charikar1}-\eqref{eq:d-ecp-edge-charikar2}.
		\eqref{eq:d-ecp-vertex-l-b} is trivial and
		\eqref{eq:d-ecp-x-l-a}-\eqref{eq:d-ecp-x-l-z} hold since
		$\mathcal{S}(U) \subseteq \mathcal{T}_C $. It is also easy to see that
		\eqref{eq:d-ecp-vertex-g-bi-yj} is satisfied since we defined $y_i$
		and $b_i$ s.t. $b_i = 1 \iff y_i = c$. Furthermore, since $y_i = c \iff
			y_i \in U$ then $ \sum^{}_{i \in V} y_i = \sum^{}_{i \in U} c = 1$
		and \eqref{eq:d-ecp-vertex-charikar1} is also satisfied.

		Let us know consider $T_k \in \mathcal{S}(U) \implies z_k = 1$. Then, by definition

		\begin{gather}
			\eta(T_k[U]) \leq \alpha \implies
			\frac{|E^{-}_{k}[U]|}{|E_{k}[U]|} \leq \alpha \\
			|E^{-}_{k}[U]| - \alpha (|E_{k}[U]|) \leq 0 \\
		\end{gather}

		Thus, the second inequality of \eqref{eq:d-ecp-alpha-constraint} is true;
		it is easy also to see that the first one is also satisfied since

		\begin{multline}
			\sum^{}_{ij \in E^{-} (T_k)} a_{ij}^{k}  - \alpha \sum^{}_{ij \in E(T_k)}
			a_{ij} ^{k} = |E^{-}_{k}[U]| - \alpha \cdot |E_{k}[U]| = \\
			(1- \alpha)|E^{-}_{k}[U]| - \alpha \cdot |E^{+}_{k}[U]| \geq -
			\alpha \cdot |E^{+}_{k}[U]| > N_k
		\end{multline}

		\eqref{eq:d-ecp-x-l-sum1}-\eqref{eq:d-ecp-x-l-sum2} are true
		since in this case ($z_k = 1$) by definition $a_{ij}^{k} = 1
			\implies x_{ij}^{k} = c$. If instead $T_k \not\in S_c$ and $z_k = 0$ we
		have that

		\begin{gather}
			\eta(T_k[U]) > \alpha \implies
			\frac{|E^{-}_{k}[U]|}{|E_{k}[U]|} > \alpha \\
			|E^{-}_{k}[U]| - \alpha (|E_{k}[U]|) > 0 \\
		\end{gather}

		and consequently the first inequality of
		\eqref{eq:d-ecp-alpha-constraint} is satisfied. Also

		\begin{multline}
			\sum^{}_{ij \in E^{-} (T_k)} a_{ij}^{k}  - \alpha \sum^{}_{ij \in E(T_k)}
			a_{ij} ^{k} = |E^{-}_{k}[U]| - \alpha \cdot |E_{k}[U]| = \\
			(1- \alpha)|E^{-}_{k}[U]| - \alpha \cdot |E^{+}_{k}[U]| \leq
			(1 -\alpha) \cdot |E^{-}_{k}[U]| < M_k
		\end{multline}

		Thus, also the second inequality is true. Also,
		\eqref{eq:d-ecp-x-l-sum1}-\eqref{eq:d-ecp-x-l-sum2} are trivially
		satisfied.

		\bigskip
		Consequently an edge contributing to $\psi(U)$ will also count in the
		objective function by $c$. So, for a given thread $T_k \in \mathcal{S}, C \in \mathcal{\hat{C}}$

		\begin{multline*}
			\sum^{}_{ij \in E^{+} (T_{k})} x_{ij} ^{k} - \sum_{ij \in E^{-}
				(T_{k})} x_{ij} ^{k} = \sum^{}_{ij \in E^{+}_k[U] } c - \sum_{ij \in E^{-}
			_k[U]} c = \\ c (|E^{+}_{k}[U]| - |E^{-}_{k}[U]|) =
			\frac{|E^{+}_{k}[U]| - |E^{-}_{k}[U]|}{|U|}
		\end{multline*}

		Thus, for each thread we have the same contribution to both $\psi(U)$
		and the objective function of
		MIP~\eqref{eq:d-ecp-objective}-\eqref{eq:d-ecp-z-k} and so the sum through
		all the threads will correspond as well.
	\end{proof}

	\begin{claim}
		Given a feasible solution of
		MIP~\eqref{eq:d-ecp-objective}-\eqref{eq:d-ecp-z-k} with value $v$ we can
		construct $U \; s.t. \; \psi(U) \geq v$.
	\end{claim}

	\begin{proof}

		Let us define $U \coloneqq \{ v_i \; s.t. \; y_i \neq 0\}$; consider
		$v_i$ s.t. $y_i \neq 0$ (if no such vertex exists then the proof is
		trivial) and let $c \coloneqq y_i$. By
		\eqref{eq:d-ecp-vertex-g-bi-yj} and \eqref{eq:d-ecp-vertex-l-b}
		we have that
		\begin{equation}
			\label{eq:d-ecpp-y-in}
			\forall v_j \in V, \; y_j \in \{ 0, c\}
		\end{equation}
		and also
		\begin{equation}
			\forall i, j \in V, \; b_i = 1 \land b_j = 1 \implies y_i = y_j = c
		\end{equation}

		Due to \eqref{eq:d-ecp-a-ij-l-bi}-\eqref{eq:d-ecp-a-ij-g-ijk} we
		have that
		\begin{equation}
			\label{eq:d-ecpp-a-iff}
			a_{ij}^{k} = 1 \iff b_i = 1 \land b_j = 1
		\end{equation}

		Now consider some $x_{ij}^{k} > 0 $. By \eqref{eq:d-ecp-x-l-a}
		$x_{ij}^{k} > 0 \implies a _{ij}^{k} = 1$ and, thanks to
		\eqref{eq:d-ecp-x-l-z}, $x_{ij}^{k} > 0 \implies z_k = 1$. Thus, combining
		this to the results in \eqref{eq:d-ecpp-y-in} and
		\eqref{eq:d-ecpp-a-iff}

		\begin{gather}
			\label{eq:d-ecpp-x-impl}
			x_{ij}^{k} > 0 \implies a_{ij}^{k} = 1 \land z_k =1 \\
			\label{eq:d-ecpp-z-impl}
			z_k = 1 \implies x_{i'j'}^{k} = c, \; \forall i', j' \in U \\
			\label{eq:d-ecpp-za-impl}
			z_k = 1 \land a_{ij}^{k} = 1 \implies x_{ij}^{k} = c
		\end{gather}

		This means that if $\exists \; x_{ij}^{k} > 0$ then all the variables
		$x_{i'j'}^{k}$ associated to edges induced by $U$ have value $c$.
		Also, since we have that $z_k = 1$ \eqref{eq:d-ecp-alpha-constraint}
		will correspond to

		\begin{equation}
			\sum^{}_{ij \in E^{-} (T_k)} a_{ij}^{k}  - \alpha \sum^{}_{ij \in E(T_k)}
			a_{ij} ^{k} \leq 0
		\end{equation}
		We showed in \eqref{eq:d-ecpp-x-impl} and \eqref{eq:d-ecpp-za-impl} that $a_{ij}^{k} = 1 \land z_k
			= 1 \iff x_{ij}^{k} > 0 $, so the edges contributing to the objective
		function are part of a \emph{non-controversial} subgraph, i.e. $T_k \in
			\mathcal{S}(U), C \in \mathcal{\hat{C}}$, so it will contribute to $\psi(U)$.

		Now suppose $\exists \; T_k \in \mathcal{S}(U), \; C \in \mathcal{\hat{C}},
			\; z_k = 0$. By definition of $\mathcal{S}(U)$ we have that

		\begin{equation}
			\eta(T_k[U]) \leq \alpha  \implies \sum^{}_{ij \in E^{-} (T_k)} a_{ij}^{k}  - \alpha \sum^{}_{ij \in E(T_k)}
			a_{ij} ^{k} \leq 0
		\end{equation}

		because $a_{ij}^{k} = 1$ for all edges induced by $U$. But, if $z_k =
			0$ then constraint \eqref{eq:d-ecp-alpha-constraint} is violated
		$\implies $ \emph{contradiction}. So no such $T_k$ exists and $T_k
			\in \mathcal{S}(U) \iff z_k = 1$.

		This means that a thread contributing to the objective function of
		MIP~\eqref{eq:d-ecp-objective}-\eqref{eq:d-ecp-z-k} also counts towards
		$\psi(U)$. Due to \eqref{eq:d-ecpp-z-impl} we can then write, for
		$T_k \in \mathcal{S}(U)$

		\begin{multline*}
			\sum^{}_{ij \in E^{+} (T_{k})} x_{ij} ^{k} - \sum_{ij \in E^{-}
				(T_{k})} x_{ij} ^{k} = \sum^{}_{ij \in E^{+}_k[U] } c - \sum_{ij \in E^{-}
			_k[U]} c = \\ c (|E^{+}_{k}[U]| - |E^{-}_{k}[U]|) =
			\frac{|E^{+}_{k}[U]| - |E^{-}_{k}[U]|}{|U|}
		\end{multline*}

		So each threads equally contributes to $\psi(U)$ and the objective
		function of MIP~\eqref{eq:d-ecp-objective}-\eqref{eq:d-ecp-z-k} and
		$\psi(U) \geq v$.
	\end{proof}

	This concludes the proof of the theorem.

\end{proof}

\section{Approximation algorithms}%
\label{sub:approximation_algorithms}

We now present some approximation algorithms for solving the defined problems.
Let $\textsc{Score}_{\xi} (U)$ and $\textsc{Score}_{\psi} (U)$ be the functions computing the
\emph{Echo Chamber Score} and \emph{Densest-Echo Chamber Score} of $U$,
respectively. These subroutines iterate over the edges of the vertices in $U$,
ignoring those that are not induced by $U$, and counting for each thread $T \in
	\mathcal{T}_{C}, C \in \mathcal{\hat{C}} $ the number of edges and negative edges
to see which are \emph{controversial}, then calculating their contributions
(\autoref{alg:score_xi} shows in detail $\textsc{Score}_\xi$;
$\textsc{Score}_\psi$ can simply be computed as $\textsc{Score}_\xi(U)/|U|$).

\begin{algorithm}
	\SetAlgoLined
	\KwResult{$\xi(U)$}
	$N^{+} (T) \leftarrow 0, \; N^{-} (T) \leftarrow 0\; \forall $ threads $T
		\in \mathcal{T}_{C}, C \in \mathcal{\hat{C}}   $ \;

	\ForEach{$v_{i} \in U$}{
		$S_i \leftarrow$ edges starting from $v_{i} $ \;
		\ForEach{$e_{ij} \in S_i$ \textbf{if} $v_{j} \in U$}{
			$T_{ij}  \leftarrow$ thread of $e_{ij} $ \;
			$w_{ij}  \leftarrow$ weight of $e_{ij} $ \;

			\uIf{$w_{ij} \geq 0$}{
				$N^{+}(T_{ij} ) \leftarrow N^{+}(T_{ij} ) + 1$ \;
			}\Else{
				$N^{-}(T_{ij} ) \leftarrow N^{-}(T_{ij} ) + 1$ \;
			}

		}
	}

	$\xi(U) \leftarrow 0$ \;
	$\eta(T) \leftarrow\frac{N^{-}(T)}{(N^{-}(T) + N^{+} (T))} \; \forall $ threads $T
		\in \mathcal{T}_{C}, C \in \mathcal{\hat{C}}   $  \;
	\ForEach{$T\in \mathcal{T}_{C}, C \in \mathcal{\hat{C}}$ \textbf{if}
		$ \eta(T) \leq \alpha $}{
		$\xi(U) \leftarrow \xi(U) + N^{+}(T) - N^{-}(T)$
	}
	\Return $\xi(U)$ \;

	\caption{The $\textsc{Score}_{\xi}  $ subroutine}
	\label{alg:score_xi}
\end{algorithm}

\subsection{The $\beta$ algorithm}%
\label{ssub:the_beta_approach}

This algorithm (\autoref{alg:algorithm_beta}) construct a set of users $U$ by
iteratively adding the node which increases the most the score or removing from
$U$ the one which contributes the least, stopping when the score cannot be
increased by adding a node. Frequency of addition and removal are regulated
through $\beta $ (for smaller values a higher density is to be expected,
generally).

\begin{algorithm}
	\SetAlgoLined
	% \KwResult{Write here the result }
	$U = \{$ random node $\}$\;
	$\xi(U) = 0$ \;
	\While{ $\exists \; v_{j} \; s.t. \; \textsc{Score}_{\xi} (U \bigcup \; \{
			v_{j} \} )> \xi(U)$}{
		$N(U) \leftarrow $ neighbours of vertices in $U$ in the graph $G$ \;
		With probability $\beta $\:  {
			$U \leftarrow U \bigcup \; \{ \arg\max_{v_{j} \in N(U)}
				\textsc{Score}_{\xi} (U \bigcup \;
				\{ v_{j} \}) \}$ \;
		}

		With probability $(1 - \beta )$ \: {
			$U \leftarrow U \setminus \{ \arg\max_{v_{j} \in U }
				\textsc{Score}_{\xi} (U \setminus \{ v_{j} \}) \}$ \;
		}

		% \uIf{1 is sampled from $Be(\beta )$}{
		%     $v = \arg\max_{v_{j} \in V \setminus U }
		%         \textsc{Score}_{\xi} (U \bigcup \;
		%         \{ v_{j} \}) $ \;
		%     $U \leftarrow U \bigcup \; \{ v \}$ \;
		% }\Else{
		%     $v =  \arg\max_{v_{j} \in U }
		%         \textsc{Score}_{\xi} (U \setminus \{ v_{j} \}) $ \;
		%     $U \leftarrow U \setminus \; \{ v \}$ \;
		% }
	}
	\Return $\textsc{Score}_{\xi} (U)$ \;
	\caption{$\beta$ algorithm}
	\label{alg:algorithm_beta}
\end{algorithm}

In addition, one may also want to ignore a node when it is removed for the
next iterations, in order to avoid stucking the algorithm in repeatedly adding
and taking out from $U$ the same vertex.

The result is clearly dependent on the choice of the initial node: the process
should be repeated for different initial nodes; also, a variant
of the algorithm prefers starting from the vertices with the highest fraction of positive edges.

One of the limitations of this approach is that the algorithm will only find
sets of users that are connected in the original graph.

\subsection{Peeling algorithm}%
\label{ssub:peeling_algorithm}

Inspired to the greedy algorithm proposed in \cite{charikar2000greedy}, these
algorithms starts by considering as set of $U = V$, all the vertices,
iteratively removing the worst nodes (\autoref{alg:algorithm_peeling})

\begin{algorithm}
	\SetAlgoLined
	% \KwResult{Write here the result }
	$U = V$\;
	$S = \textsc{Score}_{\xi}(U)$ \;
	\While{$U \neq \emptyset$ }{
		$v = \arg\max_{v_{j} \in U }
			\textsc{Score}_{\xi} (U \setminus \{ v_{j} \})$ \; $U \leftarrow U \setminus \{ v \}$ \;

		$S_i = \textsc{Score}_{\xi} (U) $ \;
	}
	\Return $\arg\max_{i}S_i$ \;

	\caption{Peeling algorithm}
	\label{alg:algorithm_peeling}
\end{algorithm}

If many nodes produce the same score, then one of them is
randomly selected (or, alternatevily, the one which has the highest fraction of
negative edges).

\subsection{Rounding algorithm}%
\label{ssub:rounding_algorithm}

This algorithm reconstructs a solution starting from the results of the
relaxation of the exact models and is again inspired by the algorithm for
Reconstructing the exact solution from the \acrshort{LP} model in
\cite{charikar2000greedy}.

Let $r_{i}$ and $r_{ij} ^{k} $ be the value of $y_i$ and $x_{ij}^{k} $ in the
solution of the relaxation, respectively.

Let $\tilde{E}$ be the sequence of edges ordered in ascending order by $r_{ij}
		^{k} $. The algorithm (\autoref{alg:algorithm_rounding}) iterates over
the edges in $\tilde{E}$, adding them to a \emph{dummy} graph $\hat{G}$, also eventually
adding incident nodes if not already present. At each iteration it is computed
the score of the vertices in the graph $\hat{G}$ and the score
of the vertices of each component in the graph, keeping track of the best
result.

\begin{algorithm}
	\SetAlgoLined
	$\hat{G} = \leftarrow $ empty graph \;
	$\hat{V} \leftarrow $ vertices of $\hat{G}$ \;
	$S = 0$

	\ForEach{ $e_{ij}^{k} \in \tilde{E}$ }{
		$\hat{V} \leftarrow \hat{V} \bigcup \{ v_{i} \}$ \textbf{if} $v_i
			\not\in \hat{V}$ \;
		$\hat{V} \leftarrow \hat{V} \bigcup \{ v_{j} \}$ \textbf{if} $v_j
			\not\in \hat{V}$ \;

		$S \leftarrow \max(S, \; \textsc{Score}_{\xi}(\hat{V})  )$

		\ForEach{component $C$ in $\hat{G}$}{
			$S \leftarrow \max(S, \; \textsc{Score}_{\xi}(C)  )$
		}

	}

	\Return S \;
	\caption{Rounding algorithm}
	\label{alg:algorithm_rounding}
\end{algorithm}

The motivation for the algorithm can be seen in
Figures~\ref{fig:rounding-original}-\ref{fig:rounding-relaxed}: the problem
relaxation involves a solution whose value assigned to the edges can be used to
find subgraphs with many positive edges by using each separate component as set
of users $U$.

\begin{figure}
	\begin{center}
		\begin{subfigure}[b]{0.4\textwidth}
			\centering
			\tikzfig{tex/tikz/rounding_original_t1}
			\caption{$T_1$}
			\label{fig:rounding-original-t1}
		\end{subfigure}
		\begin{subfigure}[b]{0.4\textwidth}
			\centering
			\tikzfig{tex/tikz/rounding_original_t2}
			\caption{$T_2$}
			\label{fig:rounding-original-t2}
		\end{subfigure}
	\end{center}
	\caption{Example original \emph{Interaction Graph} $G$}
	\label{fig:rounding-original}
\end{figure}
\begin{figure}
	\begin{center}
		\begin{subfigure}[b]{0.4\textwidth}
			\centering
			\tikzfig{tex/tikz/rounding_integer_t1}
			\caption{$T_1$}
			\label{fig:rounding-integer-t1}
		\end{subfigure}
		\begin{subfigure}[b]{0.4\textwidth}
			\centering
			\tikzfig{tex/tikz/rounding_integer_t2}
			\caption{$T_2$}
			\label{fig:rounding-original-t2}
		\end{subfigure}
	\end{center}
	\caption{Exact solution of the example in \autoref{fig:rounding-original},
		$\alpha = 0.4$}
	\label{fig:rounding-integer}
\end{figure}
\begin{figure}
	\begin{center}
		\begin{subfigure}[b]{0.4\textwidth}
			\centering
			\scalebox{0.8}{
				\tikzfig{tex/tikz/rounding_relaxed_t1}
			}
			\caption{$T_1$, where $z_1 = 0.66$}
			\label{fig:rounding-relaxed-t1}
		\end{subfigure}
		\begin{subfigure}[b]{0.4\textwidth}
			\centering
			\scalebox{0.8}{
				\tikzfig{tex/tikz/rounding_relaxed_t2}
			}
			\caption{$T_2$, where $z_2 = 1.0$}
			\label{fig:rounding-relaxed-t2}
		\end{subfigure}
	\end{center}
	\caption{Solution of the relaxation of $G$ of
		\autoref{fig:rounding-original}, $\alpha = 0.4$}
	\label{fig:rounding-relaxed}
\end{figure}

While one may think from these examples that the relaxation trivially assignes
non-zero values only to positive edges, \autoref{fig:rounding-original2} shows
a case in which a negative edge, $e_{31}$, gets the value of $1$; furthermore,
in this situation the algorithm is able to reconstruct the exact solution of the problem.

\begin{figure}
	\centering
	\tikzfig{tex/tikz/rounding_original2}
	\caption[Example of rounding algorithm finding the exact solution]{Another
		\emph{Interaction graph} example, with only one thread. In
		this case the rounding algorithm is able to find the exact solution by
		selecting all the nodes except for $v_4$. In the result of the
		relaxation all the edges except for $e_{42}$ get the value of $1$}%
	\label{fig:rounding-original2}
\end{figure}

\section{Alternative formulations}%
\label{sec:alternative-formulations}

Due to the intrinsic complexity of the problems
(\autoref{ch:complexity}) we define variants of
the \acrshort{ECP} and \acrshort{D-ECP} problems, for some of which we are also able to find an exact solution.

For these new problems we need to define new graphs, obtained by preprocessing
the \emph{Interaction Graph}.

\subsection{The \acrlong{PA} Graph}%
\label{sub:pa-graph}

Let $G = (V, E^{+}, E^{-})$ be the \emph{interaction graph}, $\delta(v_{i}, v_{j})$ and
$\delta^{-} (v_{i}, v_{j})$ the sum of the edges and negative edges , respectively,
associated to controversial contents between vertices $v_{i} $ and $v_{j} $.

\bigskip

The \acrfull{PA} graph $G_P = (V_{P}, E_{P}) $ is constructed as follows from
$G$:

\begin{itemize}
	\item for any vertex $v_{i} \in V$ add a corresponding vertex in $V_{P} $
	\item for any pair of vertices $v_i, v_j$ in $G$ let $\eta(v_i,v_j)
		      \coloneqq \frac{\delta^{-} (v_i,v_j)}{\delta (v_i,v_j)} $. If
	      $\eta(v_i,v_j) \leq \alpha $ add a positive edge between $v_{i} $ and
	      $v_{j} $ in $G_{P} $ \footnotemark.
	      % \item don't add any edge otherwise between $v_{i} $ and $v_{j} $
\end{itemize}

\footnotetext{If, instead, $\eta(v_i,v_j) > \alpha $ or $\delta(v_{i}, v_{j}) =
		0$ then don't add any edge between the 2 vertices}



The problem then is finding the Densest Subgraph of $G_P$, i.e., if $E_{P} [U]$
is the set of edges induced on $G_P$ by $U \subseteq V$, finding $U$ maximizing

\begin{equation}
	\xi(U) = \frac{|E_{P} [U]|}{|U|}
\end{equation}

\subsection{The \acrlong{TPA} Graph}%
\label{ssub:the_tpa_graph}

Differently from the previous method, in this case edges are aggregated
separately for each thread.

\bigskip

More specifically, given an \emph{interaction graph} $G = (V, E^{+}, E^{-})$ ,
let $\delta_{T}(v_{i}, v_{j})$ and
$\delta^{-} _{T}(v_{i}, v_{j})$ the sum of the edges and negative edges , respectively,
associated to thread $T$ between vertices $v_{i} $ and $v_{j} $, being $T \in
	\mathcal{T}_{C}, C \in \mathcal{\hat{C}}$.

The construction of the \acrfull{TPA} Graph $G_{TP} = (V_{TP}, E_{TP})$ goes as follows

\begin{itemize}
	\item for any vertex $v_{i} \in V$ add a corresponding vertex in $V_{P} $
	\item for any thread $T \in
		      \mathcal{T}_{C}, C \in \mathcal{\hat{C}}$ and pair of vertices
	      $v_i, v_j$ in $G$ let $\eta_{T}(v_i,v_j)
		      \coloneqq \frac{\delta^{-} _{T}(v_i,v_j)}{\delta _{T}(v_i,v_j)} $. If
	      $\eta_{T}(v_i,v_j) \leq \alpha $ add a positive edge between $v_{i} $ and
	      $v_{j} $ in $G_{TP} $, in the layer associated to thread $T$.
	      % \item don't add any edge otherwise between $v_{i} $ and $v_{j} $
\end{itemize}

We can then solve on $G_{TP}$

\begin{enumerate}
	\item the Densest Subgraph Problem (or, equivalently, the
	      \acrshort{DCS}-MM)
	\item the \acrshort{O2BFF} Problem, more specifically \acrshort{O2BFF}-AM
\end{enumerate}

% \subsection{Validity of method}
% \todo[inline]{How will you know if your results are valid?}

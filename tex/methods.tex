\chapter{Methods}
\label{ch:methods}

This chapter focuses on describing the main topic of the
research: the problems are motivated and defined in
\autoref{sec:the_echo_chamber_problem}, and their complexity is then analyzed
in \autoref{sec:problem_complexity_and_approximability}, while proposing some
alternative problem definitions. Algorithms and models for solving and
approximating the problems are then proposed in
\autoref{sec:solving_the_problem}; the definition of a generative model
concludes the chapter (\autoref{sec:generative_model}).

\section{The \acrlong{ECP}}%
\label{sec:the_echo_chamber_problem}

This section starts by outlining the graph on which the analysis is carried
out (\autoref{sub:interaction-graph}), the process for building it
(\autoref{sub:collection_and_preprocessing}) and finally defines the
\acrlong{ECP} and the \acrlong{D-ECP} (\autoref{sub:the_problem_definition} and
\autoref{sub:the_densest_echo_chamber_problem}).

\subsection{The Interaction Graph}
\label{sub:interaction-graph}

The \emph{Interaction Graph} $G$ is the graph which encodes the informations
regarding the interactions between the users.

In this graph each users is associated to a vertex $v \in V$ and each edge to
an interaction between the $2$ corresponding users it links. For this reason we
will sometime refer to vertices as users in the rest of the document.

$G = (V, E^{+}, E^{-})  $ is a
signed and weighted graph, the weights being in the interval $[-1, +1]$,
corresponding to positive and negative interactions, meaning that for smaller
the value of the weight the interaction will more "negative".

The \emph{Interaction Graph} is also directed, so that an edge from vertex $v_{i}
$ to vertex $v_{j} $ corresponds to a reply from user $v_{i} $ to $v_{j} $.

Let a content $C$ be any kind of resource that triggers a discussion in one or
more threads $T$. The set of threads associated to $C$ is denoted as
$\mathcal{T}_{C} $. A content is usually represented by a newspaper article and
it is identified by its url, e.g.

	{\footnotesize
		\begin{center}
			\url{https://www.nytimes.com/2021/03/04/us/richard-barnett-pelosi-tantrum.html}
		\end{center}
	}

A corresponding thread may then be, for example, a user posting and commenting
the same url on its Twitter account (see \autoref{fig:twitter-thread}), thus
generating a discussion.

\begin{figure}
	\centering
	\includegraphics[width=0.6\linewidth]{tex/img/twitter_thread.png}
	\caption{An thread associated to the mentioned New York Times article}%
	\label{fig:twitter-thread}
\end{figure}

The \emph{Interaction Graph} is a \emph{multiplex graph}, each layer being
represented by a thread $T$ whose edges are the interactions happening in it.
Note also that for this reason each of the layers can contain more than one
edge between $2$ users, as each pair of users can reply to each other more than
once.

We will also use $\mathcal{C} $ for denoting the set of contents.

An example of \emph{Interaction Graph} can be seen in
\autoref{fig:interaction-graph-example}.

\begin{figure}
	\begin{center}
		\begin{subfigure}[b]{0.3\textwidth}
			\centering
			\tikzfig{tex/tikz/graph_thread1}
			\caption{$T_{1} \in \mathcal{T}_{C_{1}} $}
			\label{fig:tex/tikz/graph_thread1.tikz}
		\end{subfigure}
		\begin{subfigure}[b]{0.3\textwidth}
			\centering
			\tikzfig{tex/tikz/graph_thread2}
			\caption{$T_{2} \in \mathcal{T}_{C_{1}} $}
			\label{fig:tex/tikz/graph_thread2.tikz}
		\end{subfigure}
		\begin{subfigure}[b]{0.3\textwidth}
			\centering
			\tikzfig{tex/tikz/graph_thread3}
			\caption{$T_{3} \in \mathcal{T}_{C_{2}} $}
			\label{fig:tex/tikz/graph_thread3.tikz}
		\end{subfigure}
	\end{center}
	\caption{An example of \emph{Interaction Graph}, green and red edges
		representing positive and negative interactions, respectively. It
		contains $3$ threads and $2$ contents, the first $2$ layers each being
		associated to a thread of content $C_{1} $, the last layer to a thread
		of content $C_{2} $}
	\label{fig:interaction-graph-example}
\end{figure}

\subsection{Collection and preprocessing}%
\label{sub:collection_and_preprocessing}

Datasets are built mainly upon $2$ social medias: Twitter and Reddit; the data
collection process, consequently, slightly differ between them.

\paragraph{Twitter}%
\label{par:twitter-data}

Twitter's \emph{Interaction Graphs} are mainly built starting from the tweets
of some important social accounts associated to well-known news source, like The
New York Times or Fox News, that tipically post links to their articles:
these profiles are taken as the source of the contents $\mathcal{C} $ of the graph.

Each time another user tweets the same url (\autoref{fig:twitter-thread}) then
it will correspond to another thread related to the same content, and all the
replies it receives will be part of this new thread.

Twitter data is retrieved with the help of Tweepy \cite{tweepy}, a Python
library for accessing the Twitter API, which has been patched for using some
features available only in the beta of the new Twitter API (v2).

\paragraph{Reddit}%
\label{par:reddit}

Differently from Twitter, Reddit focuses on subreddits, which are pages
collecting posts from different users about a specific topic (e.g. r/politics,
r/economics, $\dots$). This means that
in the datasets built from this social media the contents $\mathcal{C} $ is the
set of posts of these pages, which, differently from Twitter, most likely
come from different sources.

This posts are in turn crossposted, i.e. reposted on other subreddits. Each of
these \emph{crosspost} will eventually correspond to another thread.

The PRAW library is used for retrieving the data \cite{praw}.

\paragraph{Edge weights assignment}%
\label{par:assigning_edge_weights}

Once the threads interactions are retrieved they are passed to a state of the
art sentiment analyzer which labels them. More specifically the model used is
RoBERTa which has been adapted and retrained for dealing with Twitter
data \cite{Barbieri2020}. The model is made available by the Transformers
python library \cite{wolf-etal-2020-transformers}.

\bigskip

Finally, complying with the current privacy legislation, all the data related
to the user is pseudo-anonymized (accounts identifier are replaced by random
ones) while no data is publicly available.

\subsection{The problem definition}%
\label{sub:the_problem_definition}

The main goal of the research is finding echo chambers in social medias, more
specifically on the \emph{Interaction Graph} as defined in
\autoref{sub:interaction-graph}.

Our definition is based on the idea that echo chambers can be identified by
looking at contents which is generally highly debated (we will call
this type of content \emph{controversial}) but which is discussed with little
or no animosity in small subgraphs. This subgraphs are the \emph{Echo
	Chambers}.

\bigskip

Given an \emph{Interaction Graph} $G = (V, E^{+}, E^{-})$ on some contents
$\mathcal{C} $ and threads, let $\eta(T)$ the number of negative edges over the
total number of edges in the layer associated to thread $T$. Similarly,
let $\eta(C)$ be the fraction of negative edges in all threads associated to
content $C$.

\begin{definition}[Controversial thread]
	Let $\alpha \in [0,1]$. A thread (or content) is \emph{controversial} if
	$\eta(T) \geq \alpha$ (or, similarly, $\eta(C) \geq \alpha $). Conversely, a
	thread (or content) is \emph{non-controversial} if $\eta(T) < \alpha$ ($\eta(C) <
		\alpha$).
\end{definition}

Intutively, \emph{controversial} threads contain many negative
interactions. We denote as $\hat{\mathcal{C} } \subseteq \mathcal{C} $ the
set of \emph{controversial} contents.

\medskip

\emph{Echo Chambers} correspond to \emph{non-controversial} smaller subgraphs
(i.e. with few negative edges) discussing a
\emph{controversial} content.

More formally, let $T[U]$ the subgraph induced in the layer associated to
thread $T$ by the vertices $U$; let $|T^{+} [U]|$ and $|T^{-} [U]|$ its number
of positive and negative edges, respectively.

We define $\mathcal{S}_C (U)$ as the set of \emph{non-controversial} threads
induced by $U$, for \textit{controversial} contents, i.e.
	{\small
		\begin{equation}
			\mathcal{S} _{C} (U) = \{ T[U] \; s.t. \; T[U] \; non \;
			controversial, T \in \mathcal{T} _{C}, C
			\in \hat{\mathcal{C}}, U \subseteq V\}
		\end{equation}
	}

Thus $\mathcal{S} _C (U)$ will contain threads which are \emph{globally} non
controversial but it is defined only for contents that are \emph{globally}
controversial.

\medskip

We know define the Echo Chamber Score of a set of vertices $U$.

\begin{definition}[Echo Chamber Score]
	Let $U \subseteq V$ be a subset of vertices. Its Echo Chamber Score is

	\begin{equation}
		\label{eq:echo-chamber-score}
		\xi(U) = \sum^{}_{\mathcal{C} \in \mathcal{\hat{C}}} \sum^{}_{T[U] \in
		\mathcal{S} _{C} (U)} (|T^{+} [U]| - |T ^{-} [U]|)
	\end{equation}
\end{definition}

We can now define the \acrfull{ECP}.

\begin{problem}[\acrlong{ECP}]
Given an \emph{Interaction Graph} $G$ and $\alpha \in [0, 1]$ find a set of vertices $U \subseteq
	V$ maximizing the Echo Chamber Score (\autoref{eq:echo-chamber-score}).
\end{problem}

We will denote with $\hat{U}$ the set of users maximizing
\autoref{eq:echo-chamber-score} and with $\xi(G)$ its corresponding score, i.e.

\begin{align*}
	\hat{U} & \coloneqq \argmax_{U \subseteq V} \xi(U) & \xi(G) & \coloneqq
	\xi(\hat{U})
\end{align*}

\subsection{The Densest Echo Chamber Problem}%
\label{sub:the_densest_echo_chamber_problem}

The \acrshort{ECP} doesn't take into account the number of users producing a
certain score; this means that the set $U$ may involve disconnected and sparse
subgraphs, depending on the structure of the graph $G$.

For this reason it is interesting also to study another variant of the
\acrshort{ECP}, the \acrlong{D-ECP}, which we now define.

\begin{definition}[Echo Chamber Score]
	Let $U \subseteq V$ be a subset of vertices. Its Densest Echo Chamber Score is

	\begin{equation}
		\label{eq:densest-echo-chamber-score}
		\xi(U) = \sum^{}_{\mathcal{C} \in \mathcal{\hat{C}}} \sum^{}_{T[U] \in
		\mathcal{S} _{C} (U)} \frac{(|T^{+} [U]| - |T ^{-} [U]|)}{|U|}
	\end{equation}
\end{definition}

Similarly to the \acrshort{ECP} we can now define the corresponding problem

\begin{problem}[\acrlong{D-ECP}]
Given an \emph{Interaction Graph} $G$ and $\alpha \in [0, 1]$ find a set of vertices $U \subseteq
	V$ maximizing the Densest Echo Chamber Score (\autoref{eq:densest-echo-chamber-score}).
\end{problem}

\section{Problems complexity and approximability}%
\label{sec:problem_complexity_and_approximability}

\subsection{Hardness of ECP}%
\label{sub:ecp_is_}

\subsection{Hardness of D-ECP}%
\label{sub:ecp_is_}

\subsection{Solvable variants of the problem}%
\label{sub:solvable_variants_of_the_problem}

\section{Solving the problem}%
\label{sec:solving_the_problem}

\subsection{A MIP model for the ECP}%
\label{sub:a_mip_model_for_the_ecp}

\subsection{A MIP model for the D-ECP}%
\label{sub:a_mip_model_for_the_ecp}

\subsection{Approximation algorithms}%
\label{sub:approximation_algorithms}

\section{A generative model}%
\label{sec:generative_model}



% \subsection{Validity of method}
% \todo[inline]{How will you know if your results are valid?}

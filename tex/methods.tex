\chapter{Methods}
\label{ch:methods}

This chapter focuses on describing the main topic of the
research: the problems are motivated and defined in
\autoref{sec:the_echo_chamber_problem}, and their complexity is then analyzed
in \autoref{sec:problem_complexity_and_approximability}, while proposing some
alternative problem definitions. Algorithms and models for solving and
approximating the problems are then proposed in
\autoref{sec:solving_the_problem}; the definition of a generative model
concludes the chapter (\autoref{sec:generative_model}).

\section{The \acrlong{ECP}}%
\label{sec:the_echo_chamber_problem}

This section starts by outlining the graph on which the analysis is carried
out (\autoref{sub:interaction-graph}), the process for building it
(\autoref{sub:collection_and_preprocessing}) and finally defines the
\acrlong{ECP} and the \acrlong{D-ECP} (\autoref{sub:the_problem_definition} and
\autoref{sub:the_densest_echo_chamber_problem}).

\subsection{The Interaction Graph}
\label{sub:interaction-graph}

The \emph{Interaction Graph} $G$ is the graph which encodes the informations
regarding the interactions between the users.

In this graph each users is associated to a vertex $v \in V$ and each edge to
an interaction between the $2$ corresponding users it links. For this reason we
will sometime refer to vertices as users in the rest of the document.

$G = (V, E^{+}, E^{-})  $ is a
signed and weighted graph, the weights being in the interval $[-1, +1]$,
corresponding to positive and negative interactions, meaning that for smaller
the value of the weight the interaction will more "negative".

The \emph{Interaction Graph} is also directed, so that an edge from vertex $v_{i}
$ to vertex $v_{j} $ corresponds to a reply from user $v_{i} $ to $v_{j} $.

Let a content $C$ be any kind of resource that triggers a discussion in one or
more threads $T$. The set of threads associated to $C$ is denoted as
$\mathcal{T}_{C} $. A content is usually represented by a newspaper article and
it is identified by its url, e.g.

	{\footnotesize
		\begin{center}
			\url{https://www.nytimes.com/2021/03/04/us/richard-barnett-pelosi-tantrum.html}
		\end{center}
	}

A corresponding thread may then be, for example, a user posting and commenting
the same url on its Twitter account (see \autoref{fig:twitter-thread}), thus
generating a discussion.

\begin{figure}
	\centering
	\includegraphics[width=0.6\linewidth]{tex/img/twitter_thread.png}
	\caption{An thread associated to the mentioned New York Times article}%
	\label{fig:twitter-thread}
\end{figure}

The \emph{Interaction Graph} is a \emph{multiplex graph}, each layer being
represented by a thread $T$ whose edges are the interactions happening in it.
Note also that for this reason each of the layers can contain more than one
edge between $2$ users, as each pair of users can reply to each other more than
once.

We will also use $\mathcal{C} $ for denoting the set of contents We will assume that $G = \bigcup _{C \in \mathcal{} } $

An example of \emph{Interaction Graph} can be seen in
\autoref{fig:interaction-graph-example}.

\begin{figure}
	\begin{center}
		\begin{subfigure}[b]{0.3\textwidth}
			\centering
			\tikzfig{tex/tikz/graph_thread1}
			\caption{$T_{1} \in \mathcal{T}_{C_{1}} $}
			\label{fig:tex/tikz/graph_thread1.tikz}
		\end{subfigure}
		\begin{subfigure}[b]{0.3\textwidth}
			\centering
			\tikzfig{tex/tikz/graph_thread2}
			\caption{$T_{2} \in \mathcal{T}_{C_{1}} $}
			\label{fig:tex/tikz/graph_thread2.tikz}
		\end{subfigure}
		\begin{subfigure}[b]{0.3\textwidth}
			\centering
			\tikzfig{tex/tikz/graph_thread3}
			\caption{$T_{3} \in \mathcal{T}_{C_{2}} $}
			\label{fig:tex/tikz/graph_thread3.tikz}
		\end{subfigure}
	\end{center}
	\caption{An example of \emph{Interaction Graph}, green and red edges
		representing positive and negative interactions, respectively. It
		contains $3$ threads and $2$ contents, the first $2$ layers each being
		associated to a thread of content $C_{1} $, the last layer to a thread
		of content $C_{2} $}
	\label{fig:interaction-graph-example}
\end{figure}

\subsection{Collection and preprocessing}%
\label{sub:collection_and_preprocessing}

% todo: data collection process and compliance with current legislation

\subsection{The problem definition}%
\label{sub:the_problem_definition}

\subsection{The Densest Echo Chamber Problem}%
\label{sub:the_densest_echo_chamber_problem}

\section{Problems complexity and approximability}%
\label{sec:problem_complexity_and_approximability}

\subsection{Hardness of ECP}%
\label{sub:ecp_is_}

\subsection{Hardness of D-ECP}%
\label{sub:ecp_is_}

\subsection{Solvable variants of the problem}%
\label{sub:solvable_variants_of_the_problem}

\section{Solving the problem}%
\label{sec:solving_the_problem}

\subsection{A MIP model for the ECP}%
\label{sub:a_mip_model_for_the_ecp}

\subsection{A MIP model for the D-ECP}%
\label{sub:a_mip_model_for_the_ecp}

\subsection{Approximation algorithms}%
\label{sub:approximation_algorithms}

\section{A generative model}%
\label{sec:generative_model}



% \subsection{Validity of method}
% \todo[inline]{How will you know if your results are valid?}

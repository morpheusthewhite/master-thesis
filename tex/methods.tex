\chapter{Methods}
\label{ch:methods}

This chapter focuses on describing the main topic of the
research: the problems are motivated and defined in
\autoref{sec:the_echo_chamber_problem}, and their complexity is then analyzed
in \autoref{sec:problem_complexity_and_approximability}, while proposing some
alternative problem definitions. Algorithms and models for solving and
approximating the problems are then proposed in
\autoref{sec:solving_the_problem}; the definition of a generative model
concludes the chapter (\autoref{sec:generative_model}).

% \section{The \acrlong{ECP}}%
% \label{sec:the_echo_chamber_problem}

% This section starts by outlining the graph on which the analysis is carried
% out (\autoref{sub:interaction-graph}), the process for building it
% (\autoref{sub:collection_and_preprocessing}) and finally defines the
% \acrlong{ECP} and the \acrlong{D-ECP} (\autoref{sub:the_problem_definition} and
% \autoref{sub:the_densest_echo_chamber_problem}).

% \subsection{Collection and preprocessing}%
% \label{sub:collection_and_preprocessing}
%
% Datasets are built mainly upon $2$ social medias: Twitter and Reddit; the data
% collection process, consequently, slightly differ between them.
%
% \paragraph{Twitter}%
% \label{par:twitter-data}
%
% Twitter's \emph{Interaction Graphs} are mainly built starting from the tweets
% of some important social accounts associated to well-known news source, like The
% New York Times or Fox News, that tipically post links to their articles:
% these profiles are taken as the source of the contents $\mathcal{C} $ of the graph.
%
% Each time another user tweets the same url (\autoref{fig:twitter-thread}) then
% it will correspond to another thread related to the same content, and all the
% replies it receives will be part of this new thread.
%
% Twitter data is retrieved with the help of Tweepy \cite{tweepy}, a Python
% library for accessing the Twitter API, which has been patched for using some
% features available only in the beta of the new Twitter API (v2).
%
% \paragraph{Reddit}%
% \label{par:reddit}
%
% Differently from Twitter, Reddit focuses on subreddits, which are pages
% collecting posts from different users about a specific topic (e.g. r/politics,
% r/economics, $\dots$). This means that
% in the datasets built from this social media the contents $\mathcal{C} $ is the
% set of posts of these pages, which, differently from Twitter, most likely
% come from different sources.
%
% This posts are in turn crossposted, i.e. reposted on other subreddits. Each of
% these \emph{crosspost} will eventually correspond to another thread.
%
% The PRAW library is used for retrieving the data \cite{praw}.
%
% \paragraph{Edge weights assignment}%
% \label{par:assigning_edge_weights}
%
% Once the threads interactions are retrieved they are passed to a state of the
% art sentiment analyzer which labels them. More specifically the model used is
% RoBERTa which has been adapted and retrained for dealing with Twitter
% data \cite{Barbieri2020}. The model is made available by the Transformers
% python library \cite{wolf-etal-2020-transformers}.
%
% \bigskip
%
% Finally, complying with the current privacy legislation, all the data related
% to the user is pseudo-anonymized (accounts identifier are replaced by random
% ones) while no data is publicly available.

\section{An exact solution for the problems}%
\label{sec:solving_the_problem}

Now we present some techniques for calculating exactly and approximating both
the \acrshort{ECP} and the \acrshort{D-ECP}.

\subsection{A MIP model for the \acrshort{ECP}}%
\label{sub:a_mip_model_for_the_ecp}

Let $G$ be our \emph{Interaction Graph} for contents $\mathcal{C} $,
controversial contents $\mathcal{\hat{C}} \subseteq \mathcal{C} $ and
threads $T \in \mathcal{T}_{C}, \; C \in \mathcal{C} $. Let $E_k$ bet the set
of all edges of thread $T_k$ associated to a controversial content, i.e. $T_{k}
	\in \mathcal{T}_{C}, C \in \mathcal{\hat{C}}$; let also $E^{+}_k $
and $E^{-}_k $ be the set its positive and negative edges, respectively. Fix $\alpha \in [0, 1]$.

The following \acrshort{MIP} model is able to solve the \acrshort{ECP} on $G$
for values of $\alpha \leq 0.5$.

\begin{equation}
	\label{eq:ecp-exact1}
	\text{maximize} \; \sum_{ T_{k} \in \mathcal{T}_{C}, \; C \in
		\mathcal{\hat{C}} } \big( \sum^{}_{ij \in E^{+} (T_{k})} x_{ij}
		^{k} - \sum_{ij \in E^{-} (T_{k})} x_{ij} ^{k} \big)
\end{equation} \begin{center} subject to \end{center}
\begin{gather}
	\label{eq:ecp-v1}
	x _{ij}^{k}  \leq y_i \quad\quad \forall ij \in E_k  \\
	\label{eq:ecp-v2}
	x _{ij}^{k}  \leq y_j \quad\quad \forall ij \in E_k \\
	\label{eq:ecp-t1}
	x _{ij}^{k}  \leq z_k \quad\quad \forall ij \in E_k \\
	\label{eq:ecp-e1}
	x _{ij} ^{k} \geq - 2 + y_i + y_j + z_k \quad\quad \forall ij \in E_k \\
	\label{eq:ecp-alpha-constraint1}
	\sum^{}_{ij \in E_k^{-} } x_{ij}^{k}  - \alpha \sum^{}_{ij \in E_k}
	x_{ij} ^{k}  \leq 0 \quad\quad \forall T_{k} \in \mathcal{T} _{C}, C \in
	\hat{\mathcal{C}} \\
	\label{eq:ecp-vertex-def1}
	y _{i} \in  \{0, 1\} \quad\quad \forall i \in V \\
	\label{eq:ecp-edge-def1}
	0 \leq x _{ij} ^{k}  \leq 1 \quad\quad \forall ij \in E_k \\
	\label{eq:ecp-thread-def1}
	0 \leq z _{k} \leq 1 \quad\quad \forall T_{k} \in \mathcal{T} _{C}, C \in
	\hat{\mathcal{C}}
\end{gather}

We will know show that the \acrshort{ECP} can be solved through
MIP~\ref{eq:ecp-exact1}-\ref{eq:ecp-thread-def1}.

\begin{theorem}
	\label{th:ecp-mip}
	Let $G = (V, E^{+}, E^{-})$ be an \emph{Interaction Graph} and $\alpha \in
		[0, 0.5]$.

	\begin{equation}
		\max_{U \subseteq V} \xi(U) = \text{OPT(MIP)}
	\end{equation}

	where $OPT(MIP)$ denotes the optimal solution to
	MIP~\ref{eq:ecp-exact1}-\ref{eq:ecp-thread-def1}.
\end{theorem}

\begin{proof}
	We will show the equality by first proving that $RHS \geq LHS$ and then
	that $LHS \geq RHS$.

	\begin{claim}
		For any $U \subseteq V$, the
		MIP~\ref{eq:ecp-exact1}-\ref{eq:ecp-thread-def1} gets value $\geq \xi(U)$.
	\end{claim}

	\begin{proof}
		Let $E_k[U]$ the set of edges induced by $U$ in thread $T_k$. We construct a solution as follows

		\begin{gather}
			\label{eq:ecpp-y}
			y_i = \begin{cases}
				1 & \text{if } v_{i} \in U \\
				0 & \text{otherwise}
			\end{cases} \\
			\label{eq:ecpp-z}
			z_k = \begin{cases}
				1 & \text{if } T_{k} \in S_{C}(U), \; C \in \mathcal{\hat{C}} \\
				0 & \text{otherwise}
			\end{cases} \\
			\label{eq:ecpp-x}
			x_{ij}^{k} = \begin{cases}
				1 & \text{if } e_{ij} \in E_{k}[U], \; T_{k} \in S_C(U), \; C \in
				\mathcal{\hat{C}}                                                 \\
				0 & \text{otherwise}
			\end{cases}
		\end{gather}

		To satisfy Equations~\ref{eq:ecp-v1}-\ref{eq:ecp-e1} we need that
		\begin{equation}
			\label{eq:ecpp-x-iff}
			x_{ij}^{k} = 1 \iff y_i = 1 \land y_j = 1 \land z_k = 1
		\end{equation}

		Which is always true since we defined $x_{ij}^{k}$ to be $1$ only if it
		is associated to an edge induced in a $T_k \in S_{C}(U), \; C \in
			\mathcal{\hat{C}} $.

		Let us now consider a thread $T_k \in S_C(U)$; this means that

		\begin{equation}
			\eta(T[U]) \leq \alpha \implies \frac{|E^{-}_{k}[U]|}{|E_{k}[U]|} \leq
			\alpha \implies |E^{-}_{k}[U]| - \alpha |E_{k}[U]| \leq 0
		\end{equation}

		so \autoref{eq:ecp-alpha-constraint1} is satisfied. It is easy to see that
		if $T_k \not\in S_C(U)$ then $x_{ij}^{k} = 0 \; \forall ij \in E_{k}$ and
		the constraint is also satisfied.

		Finally, any edge contributing to $\xi(U)$ will also
		equally contribute to the objective function.
	\end{proof}

	\begin{claim}
		Given a feasible solution of
		MIP~\ref{eq:ecp-exact1}-\ref{eq:ecp-thread-def1} with value $v$ we can
		construct $U \; s.t. \; \xi(U) \geq v$.
	\end{claim}

	\begin{proof}
		We define $U \coloneqq \{ v_{i} \; s.t. \; y_i = 1\}$. Again, by
		Equations~\ref{eq:ecp-v1}-\ref{eq:ecp-e1} we have
		\autoref{eq:ecpp-x-iff}, so

		\begin{equation}
			z_k = 1 \iff x_{i'j'}^{k} = 1 \; \forall i'j'
			\in E_k[U]
		\end{equation}

		Let us know consider $T_k \; s.t. z_k = 1$ and suppose $T_k \not\in
			S_C(U), \; C \in \mathcal{\hat{C}}$. Thus $\eta(T_k[U]) \geq \alpha$,
		i.e.

		\begin{equation}
			\frac{|E^{-}_{k}[U]|}{|E_{k}[U]|} > \alpha \implies |E^{-}_{k}[U]| -
			\alpha |E_{k}[U]| > 0
		\end{equation}

		which violates \autoref{eq:ecp-alpha-constraint1}. So $z_k \implies T_k
			\in S_C(U), \; C \in \mathcal{\hat{C}}$, so $T_k[U]$ contributes to
		$\xi(U)$. So any edge contributing to the objective function equally
		contributes to $\xi(U)$. Now suppose $\exists \; T_k \in S_C(U) \; s.t.
			\; z_k = 0 \implies x_{ij}^{k} = 0$. Since $\alpha \leq 0.5$

		\begin{gather*}
			\eta(T_k[U]) \leq \alpha \implies \frac{|E^{-}_{k}[U]|}{|E_{k}[U]|}
			\leq 0.5 \\
			|E^{-}_{k}[U]| \leq 0.5 \cdot (|E^{+}_{k}[U]| + |E^{-}_{k}[U]|) \\
			0.5 \cdot |E^{+}_{k}[U]| - 0.5 |E^{-}_{k}[U]| \geq 0
			\cdot |E^{+}_{k}[U]| - |E^{-}_{k}[U]| \geq 0
		\end{gather*}

		Consequently $T_k$ will contribute positely to $\xi(U)$, so $\xi(U)
			\geq v$.
	\end{proof}

	This concludes the proof for \autoref{th:ecp-mip}.
\end{proof}

\clearpage

Let us now consider $\alpha \in (0.5, 1]$. With the current formulation if, for
a thread $T_k$, we have that $0.5 < \eta(T_k) \leq \alpha $ then the model will
prefer a solution in which all the edges $e_{ij} \in E_k$ are set to $0$, which
still satisfies all the constraints, even if the induced subgraph in the thread
is \emph{non-controversial} (and so contributes to the score and $x_{ij}^{k}  $
need to be set to 1). For this reason we need to introduce to the previous
problem the following constraints

\begin{gather}
	\label{eq:ecp-a-alpha-constraint}
	\sum^{}_{ij \in E^{-} (T_k)} a_{ij}^{k}  - \alpha \sum^{}_{ij \in E(T_k)}
	a_{ij} ^{k} \geq - N_{k} z_{k}  \quad\quad \forall T_{k} \in \mathcal{T} _{C}, C \in
	\hat{\mathcal{C}} \\
	\label{eq:ecp-a-ij-g-i-j}
	a_{ij}^{k} \geq -1 + y_i + y_j \quad\quad \forall ij \in E_k \\
	\label{eq:ecp-a-ij-l-i}
	a_{ij}^{k} \leq y_i\quad\quad \forall ij \in E_k \\
	\label{eq:ecp-a-ij-l-j}
	a_{ij}^{k} \leq y_j \quad\quad \forall ij \in E_k \\
	\label{eq:ecp-a-domain-2}
	0 \leq a_{ij}^{k} \leq 1 \quad\quad \forall ij \in E_k \\
	\label{eq:ecp-z-domain-2}
	0 \leq z _{k} \in  \{0, 1\} \quad\quad \forall T_{k} \in \mathcal{T} _{C}, C \in
	\hat{\mathcal{C}}
\end{gather}

Where \autoref{eq:ecp-z-domain-2} replaces \autoref{eq:ecp-thread-def1} and
$N_k \coloneqq \alpha |E^{+}_k| $ is a positive constant (its value correspond
to the minimum possible value achievable by the LHS to make the constraint as
tight as possible).

Variables $a_{ij}^{k}  $, associated to edges, are set to $1$ iff the
corresponding vertices are in $U$
(Equations~\ref{eq:ecp-a-ij-g-i-j}-\ref{eq:ecp-a-ij-l-j}). Consequently, if the
subgraph induced is not controversial then by definition the LHS of
\autoref{eq:ecp-a-alpha-constraint} will be smaller than $0$, forcing $z_k$ to
be 1, thus bounding the corresponding $x_{ij} ^{k} $ to be $1$ due to
\autoref{eq:ecp-e1}.


\subsection{A MIP model for the \acrshort{D-ECP}}%
\label{sub:a_mip_model_for_the_d_ecp}

Similarly to the \acrshort{ECP}, here we propose a \acrshort{MIP} model for
finding a solution for the \acrshort{D-ECP}, $\alpha \in [0, 0.5]$.

For simplifying notation we define $E_{k} \coloneqq E(T_{k}), T_{k} \in
	\mathcal{T}_{C}, C \in \mathcal{\hat{C}}$.

\begin{equation}
	\text{maximize} \; \sum_{ T_{k} \in \mathcal{T}_{C}, \; C \in
		\mathcal{\hat{C}} } \big( \sum^{}_{ij \in E^{+} (T_{k})} x_{ij}
		^{k} - \sum_{ij \in E^{-} (T_{k})} x_{ij} ^{k} \big)
\end{equation}
\begin{gather}
	\label{eq:d-ecp-a-ij-l-bi}
	a_{ij}^{k} \leq b_{i} \quad\quad \forall ij \in E_k \\
	\label{eq:d-ecp-a-ij-l-bj}
	a_{ij}^{k} \leq b_{j} \quad\quad \forall ij \in E_k \\
	\label{eq:d-ecp-a-ij-g-ijk}
	a _{ij} ^{k} \geq - 1 + b_i + b_j \quad\quad \forall ij \in E_k \\
	\label{eq:d-ecp-alpha-constraint}
	-N_{k} z_k \leq \sum^{}_{ij \in E^{-} (T_k)} a_{ij}^{k}  - \alpha \sum^{}_{ij \in E(T_k)}
	a_{ij} ^{k}  \leq 0 \quad\quad \forall T_{k} \in \mathcal{T} _{C}, C \in
	\hat{\mathcal{C}} \\
	\label{eq:d-ecp-edge-charikar1}
	x _{ij}^{k}  \leq y_i \quad\quad \forall ij \in E_{k} \\
	\label{eq:d-ecp-edge-charikar2}
	x _{ij} ^{k} \leq y_j \quad\quad \forall ij \in E_k \\
	\label{eq:d-ecp-vertex-charikar1}
	\sum^{}_{i \in V} y_i \leq 1 \\
	\label{eq:d-ecp-vertex-l-b}
	y_i \leq b_i \quad\quad \forall i \in V \\
	\label{eq:d-ecp-vertex-g-bi-yj}
	y_i \geq -1 + b_i + y_j \quad\quad \forall i,j \in V \\
	\label{eq:d-ecp-}
	x_{ij}^{k} \geq -2 + a_{ij} ^{k} + z_k + y_i \quad \forall ij \in E_k \\
	\label{eq:d-ecp-}
	x_{ij}^{k} \geq -2 + a_{ij} ^{k} + z_k + y_j \quad \forall ij \in E_k \\
	\label{eq:d-ecp-}
	x_{ij} ^{k} \leq a_{ij} ^{k} \quad\quad \forall ij \in E_k \\
	\label{eq:d-ecp-a-ij}
	a _{ij} ^{k}  \in \{0, 1\} \quad\quad \forall ij \in E_k\\
	\label{eq:d-ecp-b-i}
	b _{i} \in \{0, 1\} \quad\quad \forall i \in V \\
	\label{eq:d-ecp-y-i}
	y _{i} \geq 0 \quad\quad \forall i \in V \\
	\label{eq:d-ecp-x-ij}
	x _{ij} ^{k}  \geq 0 \quad\quad \forall ij \in E_k\\
	\label{eq:d-ecp-z-k}
	0 \leq z _{k} \leq 1 \quad\quad \forall T_{k} \in \mathcal{T} _{C}, C \in
	\hat{\mathcal{C}}
\end{gather}
\begin{itemize}
	\item $y_i$ and $b_i$ variables are associated to vertices. $b_i$ are used
	      to identify nodes which are in $U$, similarly to
	      \autoref{sub:a_mip_model_for_the_ecp}. Thus we define $U \coloneqq \{ v_i
		      \; s.t. \; y_i \neq 0\}$.

	      $y_i$ variables, similarly to
	      \autoref{eq:charikar-model-densest-subgraph}, are needed for
	      computing the density value, the objective function,
	      by forcing the edges of the node $v_{i} $ to take a value lower that
	      $y_i$ (\autoref{eq:d-ecp-edge-charikar1} and
	      \autoref{eq:d-ecp-edge-charikar2}).
	      Also, thanks to \autoref{eq:d-ecp-vertex-g-bi-yj} all the $y_i$ will either take
	      value $0$ or equal to $y_j \neq 0$.
	\item $x_{ij}^{k}  $ and $a_{ij}^{k}  $ are associated to edges: the
	      first one will assume the value of the edge contribution to the score (in
	      our case $0$ or $1/|U|$) while the second is a binary variable taking
	      value $1$ if the edge is contributing to the score, $0$ otherwise.
	\item $z_k$ are the exact analogous of its homonym in
	      \autoref{sub:a_mip_model_for_the_ecp}.
\end{itemize}

In Equations~\ref{eq:d-ecp-a-ij-l-bi}-\ref{eq:d-ecp-alpha-constraint}
we can easily recognize the same constraints as in the \acrshort{ECP}, while in
Equations~\ref{eq:d-ecp-edge-charikar1}-\ref{eq:d-ecp-vertex-charikar1}
Charikar's model for solving the Densest Subgraph \cite{charikar2000greedy}.



\section{Approximation algorithms}%
\label{sub:approximation_algorithms}

We now present some approximation algorithms for solving the defined scores.
Let $\textsc{Score}_{\xi} (U)$ and $\textsc{Score}_{\psi} (U)$ be the functions computing the
\emph{Echo Chamber Score} and \emph{Densest-Echo Chamber Score} of $U$,
respectively. These subroutines iterate over the edges of the vertices in $U$,
ignoring those that are not induced by $U$, and counting for each thread $T \in
	\mathcal{T}_{C}, C \in \mathcal{\hat{C}} $ the number of edges and negative edges
to see which are \emph{controversial}, then calculating their contributions
(\autoref{alg:score_xi} shows in detail $\textsc{Score}_\xi$;
$\textsc{Score}_\psi$ can simply be computed as $\xi(U)/|U|$).

\begin{algorithm}
	\SetAlgoLined
	\KwResult{$\xi(U)$}
	$N^{+} (T) \leftarrow 0, \; N^{-} (T) \leftarrow 0\; \forall $ threads $T
		\in \mathcal{T}_{C}, C \in \mathcal{\hat{C}}   $ \;

	\ForEach{$v_{i} \in U$}{
		$S_i \leftarrow$ edges starting from $v_{i} $ \;
		\ForEach{$e_{ij} \in S_i$ \textbf{if} $v_{j} \in U$}{
			$T_{ij}  \leftarrow$ thread of $e_{ij} $ \;
			$w_{ij}  \leftarrow$ weight of $e_{ij} $ \;

			\uIf{$w_{ij} \geq 0$}{
				$N^{+}(T_{ij} ) \leftarrow N^{+}(T_{ij} ) + 1$ \;
			}\Else{
				$N^{-}(T_{ij} ) \leftarrow N^{-}(T_{ij} ) + 1$ \;
			}

		}
	}

	$\xi(U) \leftarrow 0$ \;
	$\eta(T) \leftarrow\frac{N^{-}(T)}{(N^{-}(T) + N^{+} (T))}$ \;
	\ForEach{$T\in \mathcal{T}_{C}, C \in \mathcal{\hat{C}}$ \textbf{if}
		$ \eta(T) \leq \alpha $}{
		$\xi(U) \leftarrow \xi(U) + N^{+}(T) - N^{-}(T)$
	}

	\caption{The $\textsc{Score}_{\xi}  $ subroutine}
	\label{alg:score_xi}
\end{algorithm}

\subsection{The $\beta$ algorithm}%
\label{ssub:the_beta_approach}

This algorithm (\autoref{alg:algorithm_beta}) construct a set of users $U$ by
iteratively adding the node which increases the score the most or removing from
$U$ the one which contributes the least, stopping when the score cannot be
increased by adding a node. Frequency of addition and removal are regulated
through $\beta $ (for smaller values an higher density is to be expected,
generally).

\begin{algorithm}
	\SetAlgoLined
	% \KwResult{Write here the result }
	$U = \{$ random node $\}$\;
	$\xi(U) = \textsc{Score}_{\xi}(U)$ \;
	\While{ $\exists \; v_{j} \; s.t. \; \textsc{Score}_{\xi} (U \bigcup \; \{
			v_{j} \} )> \xi(U)$}{
		$N(U) \leftarrow $ neighbours of vertices in $U$ in the graph $G$ \;
		With probability $\beta $\:  {
			$U \leftarrow U \bigcup \; \{ \arg\max_{v_{j} \in N(U)}
				\textsc{Score}_{\xi} (U \bigcup \;
				\{ v_{j} \}) \}$ \;
		}

		With probability $(1 - \beta )$ \: {
			$U \leftarrow U \setminus \{ \arg\max_{v_{j} \in U }
				\textsc{Score}_{\xi} (U \setminus \{ v_{j} \}) \}$ \;
		}

		% \uIf{1 is sampled from $Be(\beta )$}{
		%     $v = \arg\max_{v_{j} \in V \setminus U }
		%         \textsc{Score}_{\xi} (U \bigcup \;
		%         \{ v_{j} \}) $ \;
		%     $U \leftarrow U \bigcup \; \{ v \}$ \;
		% }\Else{
		%     $v =  \arg\max_{v_{j} \in U }
		%         \textsc{Score}_{\xi} (U \setminus \{ v_{j} \}) $ \;
		%     $U \leftarrow U \setminus \; \{ v \}$ \;
		% }
	}
	\caption{$\beta$ algorithm}
	\label{alg:algorithm_beta}
\end{algorithm}

In addition, one may also want to ignore a node when it is removed for the
next iterations, in order to avoid stucking the algorithm in repeatedly adding
and taking out from $U$ the same vertex.

The result is clearly dependant on the choice of the initial node: the process
should be repeated for different initial nodes; also, a variant
of the algorithm prefers starting from the vertices with the highest fraction of positive edges.

One of the limitations of this approach is that the algorithm will only find
sets of users that are connected in the original graph.

\subsection{Peeling algorithm}%
\label{ssub:peeling_algorithm}

Inspired to the greedy algorithm proposed in \cite{charikar2000greedy}, this
algorithms starts by considering as set of $U = V$, all the vertices,
iteratively removing the worst nodes (\autoref{alg:algorithm_peeling})

\begin{algorithm}
	\SetAlgoLined
	% \KwResult{Write here the result }
	$U = V$\;
	$S = \textsc{Score}_{\xi}(U)$ \;
	\While{$U \neq \emptyset$ }{
		$v = \arg\max_{v_{j} \in U }
			\textsc{Score}_{\xi} (U \setminus \{ v_{j} \})$ \;
		$U \leftarrow U \setminus \{ v \}$ \;

		$S = \max(S, \textsc{Score}_{\xi} (U)) $ \;
	}
	\Return $S$ \;

	\caption{Peeling algorithm}
	\label{alg:algorithm_peeling}
\end{algorithm}

In the case in some iterations the algorithm is unable to choose due to the
fact that one or more nodes produce the same score, then one of them is
randomly selected (or, alternatevily, the one which has the highest fraction of
negative edges).

\subsection{Rounding algorithm}%
\label{ssub:rounding_algorithm}

This algorithm reconstruct a solution starting from the results of the
relaxation of the exact models and is again inspired by the algorithm for
Reconstructing the exact solution from the \acrshort{LP} model in
\cite{charikar2000greedy}.

Let $r_{i}$ and $r_{ij} ^{k} $ be the value of $y_i$ and $x_{ij}^{k} $ in the
solution of the relaxation, respectively.

Let $\tilde{E}$ be the sequence of edges ordered in ascending order by $r_{ij}
		^{k} $. The algorithm (\autoref{alg:algorithm_rounding}) goes by iterating over
edges in $\tilde{E}$, adding them a \emph{dummy} graph $\hat{G}$, also eventually
adding incident nodes if not already present. At each iterations it is computed
the score of the vertices that were added to the graph $\hat{G}$ and the score
of the vertices of each component in the graph, keeping track of the best
result.

\begin{algorithm}
	\SetAlgoLined
	$\hat{G} = \leftarrow $ empty graph \;
	$\hat{V} \leftarrow $ vertices of $\hat{G}$ \;
	$S = 0$

	\ForEach{ $e_{ij}^{k} \in \tilde{E}$ }{
		$\hat{V} \leftarrow \hat{V} \bigcup \{ v_{i} \}$ \textbf{if} $v_i
			\not\in \hat{V}$ \;
		$\hat{V} \leftarrow \hat{V} \bigcup \{ v_{j} \}$ \textbf{if} $v_j
			\not\in \hat{V}$ \;

		$S \leftarrow \max(S, \; \textsc{Score}_{\xi}(\hat{V})  )$

		\ForEach{component $C$ in $\hat{G}$}{
			$S \leftarrow \max(S, \; \textsc{Score}_{\xi}(C)  )$
		}

	}

	\Return S \;
	\caption{Rounding algorithm}
	\label{alg:algorithm_rounding}
\end{algorithm}

The motivation for algorithm can be seen in
Figures~\ref{fig:rounding-original}-\ref{fig:rounding-relaxed}: the problem
relaxation involves a solution whose value assigned to the edges can be used to
find subgraphs with many positive edges by using each separate component as set
of users $U$.

\begin{figure}
	\begin{center}
		\begin{subfigure}[b]{0.4\textwidth}
			\centering
			\tikzfig{tex/tikz/rounding_original_t1}
			\caption{$T_1$}
			\label{fig:rounding-original-t1}
		\end{subfigure}
		\begin{subfigure}[b]{0.4\textwidth}
			\centering
			\tikzfig{tex/tikz/rounding_original_t2}
			\caption{$T_2$}
			\label{fig:rounding-original-t2}
		\end{subfigure}
	\end{center}
	\caption{Example original \emph{Interaction Graph} $G$}
	\label{fig:rounding-original}
\end{figure}
\begin{figure}
	\begin{center}
		\begin{subfigure}[b]{0.4\textwidth}
			\centering
			\tikzfig{tex/tikz/rounding_integer_t1}
			\caption{$T_1$}
			\label{fig:rounding-integer-t1}
		\end{subfigure}
		\begin{subfigure}[b]{0.4\textwidth}
			\centering
			\tikzfig{tex/tikz/rounding_integer_t2}
			\caption{$T_2$}
			\label{fig:rounding-original-t2}
		\end{subfigure}
	\end{center}
	\caption{Exact solution of the example in \autoref{fig:rounding-original},
		$\alpha = 0.4$}
	\label{fig:rounding-integer}
\end{figure}
\begin{figure}
	\begin{center}
		\begin{subfigure}[b]{0.4\textwidth}
			\centering
			\scalebox{0.8}{
				\tikzfig{tex/tikz/rounding_relaxed_t1}
			}
			\caption{$T_1$, where $z_1 = 0.66$}
			\label{fig:rounding-relaxed-t1}
		\end{subfigure}
		\begin{subfigure}[b]{0.4\textwidth}
			\centering
			\scalebox{0.8}{
				\tikzfig{tex/tikz/rounding_relaxed_t2}
			}
			\caption{$T_2$, where $z_2 = 1.0$}
			\label{fig:rounding-relaxed-t2}
		\end{subfigure}
	\end{center}
	\caption{Solution of the relaxation of $G$ of
		\autoref{fig:rounding-original}, $\alpha = 0.4$}
	\label{fig:rounding-relaxed}
\end{figure}

While one may think from these examples that the relaxation trivially assignes
non-zero values only to positive edges, \autoref{fig:rounding-original2} shows
a case in which a negative edge, $e_{31}$ gets the value of $1$; furthermore,
in this situation the algorithm is able to reconstruct the exact solution of the problem.

\begin{figure}
	\centering
	\tikzfig{tex/tikz/rounding_original2}
	\caption[Example of rounding algorithm finding the exact solution]{Another \emph{Interaction graph} example, with a single thread. In
		this case the rounding algorithm is able to find the exact solution by
		selecting all the nodes except for $v_4$. In the result of the
		relaxation all the edges except for $e_{42}$ get the value of $1$}%
	\label{fig:rounding-original2}
\end{figure}

\section{Alternative formulations}%
\label{sec:alternative-formulations}

Due to the intrinsic complexity of the problems
(\autoref{sec:problem_complexity_and_approximability}) we define variants of
the problems, for some of which we are also able to find an exact solution.

For these new problems we need to define new graphs, obtained by preprocessing
the \emph{Interaction Graph}.

\subsection{The \acrlong{PA} Graph}%
\label{sub:pa-graph}

Let $G = (V, E^{+}, E^{-})$ be the \emph{interaction graph}, $\delta(v_{i}, v_{j})$ and
$\delta^{-} (v_{i}, v_{j})$ the sum of the edges and negative edges , respectively,
associated to controversial contents between vertices $v_{i} $ and $v_{j} $.

\bigskip

The \acrfull{PA} graph $G_P = (V_{P}, E_{P}) $ is constructed as follows from
$G$:

\begin{itemize}
	\item for any vertex $v_{i} \in V$ add a corresponding vertex in $V_{P} $
	\item for any pair of vertices $v_i, v_j$ in $G$ let $\eta(v_i,v_j)
		      \coloneqq \frac{\delta^{-} (v_i,v_j)}{\delta (v_i,v_j)} $. If
	      $\eta(v_i,v_j) \leq \alpha $ add a positive edge between $v_{i} $ and
	      $v_{j} $ in $G_{P} $ \footnotemark.
	      % \item don't add any edge otherwise between $v_{i} $ and $v_{j} $
\end{itemize}

\footnotetext{If, instead, $\eta(v_i,v_j) > \alpha $ or $\delta(v_{i}, v_{j}) =
		0$ then don't add any edge between the 2 vertices}



The problem then is finding the Densest Subgraph of $G_P$, i.e., if $E_{P} [U]$
is the set of edges induced on $G_P$ by $U \subseteq V$, finding $U$ maximizing

\begin{equation}
	\xi(U) = \frac{|E_{P} [U]|}{|U|}
\end{equation}

\subsection{The \acrlong{TPA} Graph}%
\label{ssub:the_tpa_graph}

Differently from the previous method, in this case edges are aggregated
separately for each thread.

\bigskip

More specifically, given an \emph{interaction graph} $G = (V, E^{+}, E^{-})$ ,
let $\delta_{T}(v_{i}, v_{j})$ and
$\delta^{-} _{T}(v_{i}, v_{j})$ the sum of the edges and negative edges , respectively,
associated to thread $T$ between vertices $v_{i} $ and $v_{j} $, being $T \in
	\mathcal{T}_{C}, C \in \mathcal{\hat{C}}$.

The construction of the \acrfull{TPA} Graph $G_{TP} = (V_{TP}, E_{TP})$ goes as follows

\begin{itemize}
	\item for any vertex $v_{i} \in V$ add a corresponding vertex in $V_{P} $
	\item for any thread $T \in
		      \mathcal{T}_{C}, C \in \mathcal{\hat{C}}$ and pair of vertices
	      $v_i, v_j$ in $G$ let $\eta_{T}(v_i,v_j)
		      \coloneqq \frac{\delta^{-} _{T}(v_i,v_j)}{\delta _{T}(v_i,v_j)} $. If
	      $\eta_{T}(v_i,v_j) \leq \alpha $ add a positive edge between $v_{i} $ and
	      $v_{j} $ in $G_{TP} $, in the layer associated to thread $T$.
	      % \item don't add any edge otherwise between $v_{i} $ and $v_{j} $
\end{itemize}

We can then solve on $G_{TP}$

\begin{enumerate}
	\item the Densest Subgraph Problem (or, equivalently, the
	      \acrshort{DCS}-MM)
	\item the \acrshort{O2BFF} Problem
\end{enumerate}

% \subsection{Validity of method}
% \todo[inline]{How will you know if your results are valid?}

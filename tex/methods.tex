\chapter{Method or Methods}
\todo[inline, backgroundcolor=aqua]{Metod eller Metodval}
\todo[inline]{This chapter is about Engineering-related
	content, Methodologies and Methods.  Use a self-explaining title.\\The
	contents and structure of this chapter will change with your choice of
	methodology and methods.}
\label{ch:methods}


Describe the engineering-related contents (preferably with models) and the research methodology and methods that are used in the degree project.

Give a theoretical description of the scientific or engineering methodology are you going to use and why have you chosen this method. What other methods did you consider and why did you reject them.

In this chapter, you describe what engineering-related and scientific skills you are going to apply, such as modeling, analyzing, developing, and evaluating engineering-related and scientific content. The choice of these methods should be appropriate for the problem . Additionally, you should be consciousness of aspects relating to society and ethics (if applicable). The choices should also reflect your goals and what you (or someone else) should be able to do as a result of your solution - which could not be done well before you started.

The purpose of this chapter is to provide an overview of the research method
used in this thesis. Section~\ref{sec:researchProcess} describes the research
process. Section~\ref{sec:researchParadigm} details the research
paradigm. Section~\ref{sec:dataCollection} focuses on the data collection
techniques used for this research. Section~\ref{sec:experimentalDesign}
describes the experimental design. Section~\ref{sec:assessingReliability}
explains the techniques used to evaluate the reliability and validity of the
data collected. Section~\ref{sec:plannedDataAnalysis} describes the method
used for the data analysis. Finally, Section~\ref{sec:evaluationFramework}
describes the framework selected to evaluate xxx.

\todo[inline, backgroundcolor=aqua]{Vilka vetenskapliga eller ingenjörsmetodik ska du använda och varför har du valt den här metoden. Vilka andra metoder gjorde du överväga och varför du avvisar dem.
	Vad är dina mål? (Vad ska du kunna göra som ett resultat av din lösning - vilken inte kan göras i god tid innan du började)
	Vad du ska göra? Hur? Varför? Till exempel, om du har implementerat en artefakt vad gjorde du och varför? Hur kommer ditt utvärdera den.
	Syftet med detta kapitel är att ge en översikt över forsknings metod som
	används i denna avhandling. Avsnitt~\ref{sec:researchProcess} beskriver forskningsprocessen. Avsnitt~\ref{sec:researchParadigm} detaljer forskningen paradigm. Avsnitt~\ref{sec:dataCollection} fokuserar på datainsamling
	tekniker som används för denna forskning. Avsnitt~\ref{sec:experimentalDesign} beskriver experimentell
	design. Avsnitt~\ref{sec:assessingReliability} förklarar de tekniker som används för att utvärdera
	tillförlitligheten och giltigheten av de insamlade uppgifterna. Avsnitt~\ref{sec:plannedDataAnalysis}
	beskriver den metod som används för dataanalysen. Slutligen, Avsnitt~\ref{sec:evaluationFramework}
	beskriver ramverket valts för att utvärdera xxx.
}

\begin{swedishnotes}
	Ofta kan man koppla ett antal följdfrågor till undersökningsfrågan och problemlösningen t ex
	\begin{itemize}
		\color{blue}
		\item Vilken process skall användas för konstruktion av lösningen och vilken process skall kopplas till denna för att svara på undersökningsfrågan?
		\item Hur och vilket resultat (storheter) skall presenteras både för att redovisa svar på undersökningsfrågan (resultatkapitlet i denna rapport) och redovisa resultat av problemlösningen (prototypen, ofta dokument som bilagor men vilka dokument och varför?).
		\item Vilken teori/teknik skall väljas och användas både för undersökningen (taxonomi, matematik, grafer, storheter mm)  och  problemlösning (UML, UseCases, Java mm) och varför?
		\item Vad behöver du som student leverera för att uppnå hög kvaliet (minimikrav) eller mycket hög kvalitet på examensarbetet?
		      Frågorna kopplar till de följande underkapitlen.\todo[inline, backgroundcolor=aqua]{Resonemanget bygger på att studenter på hing-programmet ofta skall konstruera något åt problemägaren och att man till detta måste koppla en intressant ingenjörsfråga. Det finns hela tiden en dualism mellan dessa aspekter i exjobbet.}
	\end{itemize}
\end{swedishnotes}

\section{Research Process}
\todo[inline, backgroundcolor=aqua]{Undersökningsrocess och utvecklingsprocess}
\label{sec:researchProcess}
Figure~\ref{fig:researchprocess} shows the steps conducted in order to carry out this research.
\begin{swedishnotes}
	Figur~\ref{fig:researchprocess} visar de steg som utförs för att genomföra

	Beskriv, gärna med ett aktivitetsdiagram (UML?), din undersökningsprocess och utvecklingsprocess.  Du måste koppla ihop det akademiska intresset (undersökningsprocess) med ursprungsproblemet (utvecklingsprocess)
	denna forskning.
	\todo[inline, backgroundcolor=aqua]{Aktivitetsdiagram från t ex UML-standard}
\end{swedishnotes}


\todo[inline, backgroundcolor=aqua]{Forskningsprocessen}

\section{Research Paradigm}
\label{sec:researchParadigm}
\todo[inline, backgroundcolor=aqua]{Undersökningsparadigm}
\begin{swedishnotes}
	Exempelvis\\
	Positivistisk (vad/hur fungerar det?) kvalitativ fallstudie med en deduktivt (förbestämd) vald ansats och ett induktivt(efterhand uppstår dataområden och data) insamlade av data och erfarenheter.
\end{swedishnotes}

\section{Data Collection}
\todo[inline]{This should also show that you are aware of the social and ethical concerns that might be relevant to your data collection method.)}
\label{sec:dataCollection}
\todo[inline, backgroundcolor=aqua]{Datainsamling}
\begin{swedishnotes}
	(Detta bör också visa att du är medveten om de sociala och etiska frågor som
	kan vara relevanta för dina data insamlingsmetod.)
\end{swedishnotes}

\subsection{Sampling}
\todo[inline, backgroundcolor=aqua]{Stickprovsundersökning}

\subsection{Sample Size}
\todo[inline, backgroundcolor=aqua]{Provstorleken}

\subsection{Target Population}
\todo[inline, backgroundcolor=aqua]{Målgruppen}

\section{Experimental design/Planned Measurements}
\label{sec:experimentalDesign}
\todo[inline, backgroundcolor=aqua]{Experimentdesign/Mätuppställning}

\subsection{Test environment/test bed/model}\todo[inline]{Describe everything that someone else would need to reproduce your test environment/test bed/model/… .}
\todo[inline, backgroundcolor=aqua]{Testmiljö/testbädd/modell}
\begin{swedishnotes}
	Beskriv allt att någon annan skulle behöva återskapa din testmiljö / testbädd / modell / …
\end{swedishnotes}

\subsection{Hardware/Software to be used}
\todo[inline, backgroundcolor=aqua]{Hårdvara / programvara som ska användas}


\section{Assessing reliability and validity of the data collected}
\todo[inline, backgroundcolor=aqua]{Bedömning av validitet och reliabilitet hos använda metoder och insamlade data }
\label{sec:assessingReliability}

\subsection{Validity of method}
\todo[inline]{How will you know if your results are valid?}
\todo[inline, backgroundcolor=aqua]{Giltigheten av metoder}
\begin{swedishnotes}
	Har dina metoder ge dig de rätta svaren och lösning? Var metoderna korrekt?
\end{swedishnotes}

\subsection{Reliability of method}
\todo[inline]{How will you know if your results are reliable?}
\todo[inline, backgroundcolor=aqua]{Tillförlitlighet av metoder}
\begin{swedishnotes}
	Hur bra är dina metoder, finns det bättre metoder? Hur kan du förbättra dem?
\end{swedishnotes}

\subsection{Data validity}
\todo[inline, backgroundcolor=aqua]{Giltigheten av uppgifter}
\begin{swedishnotes}
	Hur vet du om dina resultat är giltiga? Har ditt resultat mäta rätta?
\end{swedishnotes}

\subsection{Reliability of data}
\todo[inline, backgroundcolor=aqua]{Tillförlitlighet av data}
\begin{swedishnotes}
	Hur vet du om dina resultat är tillförlitliga? Hur bra är dina resultat?
\end{swedishnotes}


\section{Planned Data Analysis}
\todo[inline, backgroundcolor=aqua]{Metod för analys av data}
\label{sec:plannedDataAnalysis}

\subsection{Data Analysis Technique}
\todo[inline, backgroundcolor=aqua]{Dataanalys Teknik}

\subsection{Software Tools}
\todo[inline, backgroundcolor=aqua]{Mjukvaruverktyg}


\section{Evaluation framework}
\todo[inline, backgroundcolor=aqua]{Utvärdering och ramverk}
\label{sec:evaluationFramework}
\begin{swedishnotes}
	Metod för utvärdering, jämförelse mm. Kopplar till kapitel~\ref{ch:resultsAndAnalysis}.
\end{swedishnotes}

\section{System documentation}\todo[inline]{If this is going to be a complete document consider putting it in as an appendix, then just put the highlights here.}
\todo[inline, backgroundcolor=aqua]{Systemdokumentation}
\begin{swedishnotes}
	Med vilka dokument och hur skall en konstruerad prototyp dokumenteras? Detta blir ofta bilagor till rapporten och det som problemägaren till det ursprungliga problemet (industrin) ofta vill ha.\\
	Bland dessa bilagor återfinns ofta, och enligt någon angiven standard, kravdokument, arkitekturdokument, designdokumnet, implementationsdokument, driftsdokument, testprotokoll mm.
\end{swedishnotes}

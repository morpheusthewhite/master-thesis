\chapter{Results and Analysis}
\label{ch:resultsAndAnalysis}

\section{Data collection and generation}%
\label{sec:data_collection_and_generation}

We now discuss how the real-world data is retrieved and preprocessed as well as
presenting some techniques for generatic synthetic data.

\subsection{Collection and preprocessing}%
\label{sub:collection_and_preprocessing}

Datasets are built mainly upon $2$ social medias: Twitter and Reddit; the data
collection process, consequently, slightly differ between them.

\paragraph{Twitter}%
\label{par:twitter-data}

Twitter's \emph{Interaction Graphs} are mainly built starting from the tweets
of some important social accounts associated to well-known news source, like The
New York Times or Fox News, that tipically post links to their articles:
these profiles are taken as the source of the contents $\mathcal{C} $ of the graph.

Each time another user tweets the same url (\autoref{fig:twitter-thread}) then
it will correspond to another thread related to the same content, and all the
replies it receives will be part of this new thread.

Twitter data is retrieved with the help of Tweepy \cite{tweepy}, a Python
library for accessing the Twitter API, which has been patched for using some
features available only in the beta of the new Twitter API (v2).

\paragraph{Reddit}%
\label{par:reddit}

Differently from Twitter, Reddit focuses on subreddits, which are pages
collecting posts from different users about a specific topic (e.g. r/politics,
r/economics, $\dots$). This means that
in the datasets built from this social media the contents $\mathcal{C} $ is the
set of posts of these pages, which, differently from Twitter, most likely
come from different sources.

This posts are in turn crossposted, i.e. reposted on other subreddits. Each of
these \emph{crosspost} will eventually correspond to another thread.

The PRAW library is used for retrieving the data \cite{praw}.

\paragraph{Edge weights assignment}%
\label{par:assigning_edge_weights}

Once the threads interactions are retrieved they are passed to a state of the
art sentiment analyzer which labels them. More specifically the model used is
RoBERTa which has been adapted and retrained for dealing with Twitter
data \cite{Barbieri2020}. The model is made available by the Transformers
python library \cite{wolf-etal-2020-transformers}.

\bigskip

Finally, complying with the current privacy legislation, all the data related
to the user is pseudo-anonymized (accounts identifier are replaced by random
ones) while no data is publicly available.

\subsection{Synthetic data}%
\label{sub:synthetic_data}

Here we propose $2$ possible methods for generating data, the Signed SBM and
the Information spread Model.

\subsubsection{Signed SBM}%
\label{ssub:signed_sbm}

This model is very similar to the Stochastic Block Model (SBM), a model
commonly used for generating random graphs having some community structures
\cite{Newman2018}.

The Signed SBM is based on the following parameters
\begin{itemize}
	\item $b_{i} $, the group assignment of each vertex $i$
	\item $\omega ^{+} _{rs} $ and $\omega ^{-} _{rs} $, the probabilities
	      of positive and negative edges, respectively, between users in
	      group $r$ and $s$. Vertices have also a probability of not having an
	      edge, which is equal to $1 - \omega ^{-} _{rs} - \omega ^{+} _{rs} $.
	      For this reason it is clearly needed that $\omega ^{+} _{rs} + \omega ^{-} _{rs} \leq 1$.
	\item $\theta \leq 1$, controlling the reduction of probability of interacting
	      between \emph{inactive} communities
\end{itemize}

Therefore, generating a thread layer\footnote{in this model we will generate contents univocally associated to threads} for the \emph{Interaction Graph} involves the following process
\begin{enumerate}
	\item Sample $n'$ among the $n$ communities. These are the
	      \emph{active} communities in the discussion of the thread
	\item For each node pairing $i, j$ consider their corresponding groups $r$ and
	      $s$ and draw from one of the following categorical distribution to decide to add
	      a positive, negative or no edge.
	      \begin{itemize}
		      \item If both communities $r$ and $s$ are \emph{active}
		            \begin{equation*}
			            (\omega _{rs} ^{+}, \omega _{rs} ^{-}, 1 - \omega _{rs} ^{+} - \omega _{rs} ^{-})
		            \end{equation*}
		      \item otherwise
		            \begin{equation*}
			            (\theta \omega _{rs} ^{+}, \theta \omega _{rs} ^{-}, 1 - \theta (\omega _{rs} ^{+} + \omega _{rs} ^{-}))
		            \end{equation*}

	      \end{itemize}

\end{enumerate}

\subsubsection{Information spread model}%
\label{ssub:information_spread_model}

Here we describe the Information spread model, which aims at simulating the
process of informations flowing between different users of a social media.

Like in the Signed SBM, each node has a group assignment and there are probabilities of
positive and negative edges $\omega _{rs}^{+}  $ and $\omega _{rs}^{+}  $,
respectively, with $\omega ^{-} _{rs} + \omega ^{+} _{rs}
	\leq 1$. Additionally, we have the following new parameters

\begin{itemize}
	\item ${\phi_{rs} }$, the parameters of a standard SBM
	\item $\beta _a$, the probability that a node is initially activated
	\item $\beta _n$, the probability the an inactive node is activated
	      from a friend
\end{itemize}

\bigskip
Generate the \emph{follow} graph $G_{f}$ from a SBM with parameters $\{ \phi
	_{rs}  \}$; we will refer to neighbours in this graph as \emph{friends}.
The generation of a thread for the \emph{Interaction Graph} goes as follows

\begin{enumerate}
	\item Activate vertices with probability $\beta_{a}  $
	\item Any of these active nodes activates its inactive friends with
	      probability $\beta_n$
	      % \item Let $a_{i} $ be the number of \emph{active} neighbours of node
	      %     $i$ in $G$ and $m_{i} $ the number of neighbours of node $i$ in
	      %     $G$. Any node inactive from the previous step is activated with
	      %     probability $ \frac{a_i}{m_i} \beta _{n} $
	\item Similarly to before, if $2$ nodes are both active nodes they will interact according to the categorical distribution
	      \begin{equation*}
		      (\omega _{rs}
		      ^{+}, \omega _{rs} ^{-}, 1 - \omega _{rs} ^{+} - \omega _{rs} ^{-})
	      \end{equation*}

	      otherwise according to
	      \begin{equation*}
		      (\theta \omega _{rs} ^{+}, \theta \omega _{rs} ^{-}, 1
		      - \theta (\omega _{rs} ^{+} + \omega _{rs} ^{-}))
	      \end{equation*}
\end{enumerate}

\section{Major results}

\section{Validity and reliability analysis}

\section{Discussion}

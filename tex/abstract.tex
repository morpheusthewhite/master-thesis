\begin{abstract}
	\markboth{\abstractname}{}

	Social medias are becoming more and more popular and used to discuss a
	wide range of topics. On these platforms we are often experiencing
	polarization between the users, producing a clear separation between
	groups with different opinions. Echo Chambers are closely related to this
	phenomenon: an Echo Chamber is a group of users with the same alignment
	that reinforce their ideas.

	The growing complexity and quantity of online interactions require us to
	find new technique for detecting polarization and Echo Chambers. In this
	work we propose the \acrfull{ECP} and the \acrfull{D-ECP}, new formulations
	that take into account the concepts of \emph{contents} and \emph{threads}
	in finding polarization.

	Our idea is that Echo Chambers corresponds to groups of users discussing a content
	which is globally \emph{controversial}, i.e.\ triggers many negative
	interactions between the users, with no \emph{controversy}, i.e.\ with
	mainly positive interactions.

	We will show that these problems are hard to approximate within a
	non-trivial factor and propose \acrfull{MIP} models for solving and
	heuristic for approximating them. Finally, we will focus on one of these
	methods and show that it is able to find Echo Chambers in synthetic data
	but has some limitations when applied to real-world data.

	Our work sheds light on a more expressive view of the problem and its
	complexity, giving future research a new starting point for detecting
	polarization in social media and presenting some limiting problematics
	of the process.

	\ifinswedish
		\subsection*{Nyckelord}
		5-6 nyckelord\todo{Nyckelord som beskriver innehållet i uppsatsrapporten}
	\else
		\subsection*{Keywords}
		Controversy, Polarization, Echo Chambers, Social
		Networks, Signed Graphs
	\fi

\end{abstract}
\cleardoublepage

% \ifinswedish
%     \selectlanguage{english}
% \else
%     \selectlanguage{swedish}
% \fi
% \begin{abstract}
%     \markboth{\abstractname}{}
%     % todo
%
%     \subsection*{Nyckelord}
%     % todo
%
% \end{abstract}

% \cleardoublepage
% \selectlanguage{italian} \todo[inline]{Use the relevant language for abstracts for your home university.\\
%     Note that you may need to augment the set of lanaguage used in polyglossia or
%     babel. The following languages represent the languages that have been used in
%     theses at KTH in 2018-2019, except for one in Chinese.
% }
\cleardoublepage
% set to the language of the body of the thesis
\ifinswedish
	\selectlanguage{swedish}
\else
	\selectlanguage{english}
\fi

%% LyX 2.3.6 created this file.  For more info, see http://www.lyx.org/.
%% Do not edit unless you really know what you are doing.
\documentclass[english]{article}
\usepackage[T1]{fontenc}
\usepackage[latin9]{inputenc}
\usepackage[a4paper]{geometry}
\geometry{verbose}
\usepackage{babel}
\usepackage{amsmath}
\usepackage{hyperref}
% algorithms
\usepackage[ruled,vlined]{algorithm2e}

\begin{document}

\section{Echo Chamber Problem}

 (``Echo Chamber Problem'' might not be the best name, but I did
 not come up with another one.)

We propose to use our data to find echo chambers inside social networks.
The general idea is that some content is controversially discussed
inside a social network, but that inside echo chambers we do not see
this controversy. Thus, the general idea is as follows: Look at contents
in social networks that are controversially discussed \emph{globally}
and then find subgraphs in which the same contents are discussed non-controversially.
These subgraphs are the echo chambers.

We will now make this notion more formal, but it requires a few definitions.
\begin{itemize}
	\item Suppose there is a signed, directed, multi-graph $G=(V,E^{+},E^{-})$.
	      In the graph, the vertices correspond to users of a social network and
	      the edges correspond to positive/negative interactions of the different
	      users. We will call \emph{G} the \emph{interaction graph}.
	\item Consider a set of contents $\mathcal{C}$, e.g., newspaper articles
	      that are shared on a social network.
	\item For each content $C\in\mathcal{{C}}$, there exists a set of threads
	      $\mathcal{{T}}_{C}.$ Each thread is $T\in\mathcal{{T}}_{C}$ is a
	      subgraph of $G$, i.e., $T\subseteq G,$ such that in the initial post of
	      the thread a user shared the content $C$ and then other users replied
	      to this post. The edges are directed outwards from the replying user;
	      the edges have signs based on whether the interaction was
	      positive/negative.  Furthermore, we assume that the whole graph is the
	      union of the threads, i.e.,
	      $G=\bigcup_{C\in\mathcal{{C}}}\bigcup_{T\in\mathcal{{T}}_{C}}T$.
	\item Fix some $\alpha\in[0,1]$. We denote the fraction of negative edges
	      of a content $C$ and thread $T$ as $\eta(C)$ and $\eta(T)$,
	      respectively. We now define \emph{(non-)controversial}
	      threads and contents:
	      \begin{itemize}
		      \item Intuitively, a thread is controversial if there are many
		            negative interactions. Formally, we say that a thread $T$ is
		            \emph{controversial} if $\eta(T) \in [\alpha,1]$.
		            Otherwise, the thread is
		            \emph{non-controversial} (in this case, the thread contains
		            mostly positive edges).
		      \item Intuitively, a content is controversial if it triggers a
		            lot of negative interactions in the whole network. Formally,
		            we say that a content $C$ is \emph{controversial} if the
		            fraction of negative edges in
		            $\bigcup_{T\in\mathcal{{T}_{C}}}T$ is in the interval
		            $[\alpha,1]$, i.e. $\eta(C) \in [\alpha,1]$.  Otherwise,
		            the content is \emph{non-controversial}. We write
		            $\hat{\mathcal{C}} \subseteq \mathcal{C}$ to denote the set
		            of controversial contents.
	      \end{itemize}
	\item Now our intuition is that inside \emph{echo chambers} there should be
	      little controversy because users have similar opinions and there is
	      little rebuttal. In particular, echo chambers should satisfy that
	      content is which is \emph{globally} controversial is discussed in a
	      non-controversial way \emph{inside} the echo chamber.
	\item We will now formalize this notion. We will use the following
	      notation, where $U\subseteq V$ is a set of vertices:
	      \begin{itemize}
		      \item For a thread $T$, we write $T[U]$ to denote the induced subgraph
		            of $T$ only containing vertices in $U$. As above, we say
		            that $T[U]$ is \emph{controversial} if the fraction of
		            negative edges in $T[U]$ is in the interval $[\alpha,1]$.
		      \item We write $|T(U)|$ to denote the number of edges in $T(U)$
		            (irrespective of their sign).
		      \item For a controversial
		            content $C\in\hat\mathcal{{C}}$ and a set of users
		            $U\subseteq V$, we write $\mathcal{{S}}_{C}(U)$ to denote
		            all threads $T[U]$ such that $T\in\mathcal{{T}}_{C}$ and
		            $T[U]$ is non-controversial. Note that, as desired,
		            $\mathcal{{S}}_{C}(U)$ contains only threads are
		            \emph{locally} non-controversial and that
		            $\mathcal{{S}}_{C}(U)$ is only defined for contents which
		            are \emph{globally} controversial.
		      \item For a set of users $U\subseteq V$, we say that its
		            \emph{echo chamber score} $\xi(U)$ is defined as \[
			            \xi(U)=\sum_{C\in\hat\mathcal{{C}}}\sum_{T[U]\in\mathcal{{S}}_{C}(U)}|T[U]|.
		            \] Note that in the first sum we only take into account contents
		            that are globally controversial; this choice was made to ensure
		            that ``wholesome'' content has no effect on the echo chamber
		            score.
	      \end{itemize}
	\item We can now define the \textbf{Echo Chamber Problem}:
	      \begin{itemize}
		      \item \emph{Input:} An interaction graph $G=(V,E^{+},E^{-})$, a set of
		            contents $\mathcal{{C}}$, a set of threads
		            $\mathcal{{T}}_{C}$ for each $C\in\mathcal{{C}}$ and a
		            parameter $\alpha\in[0,1]$.
		      \item \emph{Goal:} Find a set of users $U\subseteq V$ that
		            maximizes the echo chamber score $\xi(U)$.
	      \end{itemize}
\end{itemize}
%

\section{Comments}
\begin{itemize}
	\item I do not insist that this is the perfect problem formulation that we
	      should study. I just think that it might be a starting point for a
	      discussion.
	\item In the echo chamber score, we could as well say that we only count
	      the number of positive edges in $T[U]$.
	\item The problem formulation has the benefit that we do not have to try to
	      infer the leaning of users towards contents.
	\item We could assume that we also have access to the \emph{follow graph},
	      i.e., the graph that encodes which user follows which other user.  In
	      that case, we could define communities based on the follow graph and
	      then check which of the communities has the highest echo chamber score.
	      We could find the communities inside the follow graph by simply using
	      existing community detection algorithms. This would also allow us to
	      check whether the echo chamber score is useful in practice without
	      having to solve the optimization problem.
	\item A problem with the above definition of the echo chamber problem that
	      it does not take into account the follow graph at all. For example,
	      suppose that in a graph all users interact with each other and half
	      them is left-wing and the other one is right-wing. Then the above
	      problem could simply pick one of the two communities (inside which
	      should be mostly positive interactions) and claim that it is an echo
	      chamber. But that is not really true because (by assumption) all users
	      interact and debate with each other regularly.
	\item We could also extend the definition of \emph{controversial} and
	      \emph{non-controversial} to the interval $[\alpha,1-\alpha]$ instead of
	      $[0,1-\alpha]$.  This would resemble the fact that there might be
	      threads such that all users from an echo chamber reply negatively. In
	      other words, there is still a clear ``majority'' reaction. This might,
	      however, come at the risk of finding opposing groups which contain a
	      lot of negative edges; this would not really go too well with the usual
	      notion of an echo chamber.
\end{itemize}

\section{Approaches}%
\label{sec:approaches}

For a graph $G$ we denote with $\xi(G)$ the maximum echo chamber score, and
$\hat U$ the corresponding set of users.

\subsection{Greedy algorithm}%
\label{sub:greedy_algorithm}

The behaviour of the algorithm depends on the choice of $\beta$, regulating
the density of the resulting set of users (for smaller values an higher density
is to be expected, generally)

\begin{algorithm}[H]
	\SetAlgoLined
	% \KwResult{Write here the result }
	$U = \{$ random node $\}$\;
	\While{$\xi(U)$ can be increased by adding a node}{
		With probability $\beta $  {
				add to $U$ the node increasing the score $\xi(U)$ the most\;
			}
		With probability $(1 - \beta )$ remove from $U$ the node contributing
		less to the score $\xi(U)$. This node will be ignored in the next iteration\;
	}
	\caption{Greedy algorithm}
\end{algorithm}

The process is repeated with different starting nodes.

A variant takes into a account the fraction of positive edges of a vertex in
sampling the initial node (the higher the fraction, the more likely it is to be
sampled).

\subsection{Integer Linear Programming model}%
\label{sub:integer_linear_programming_model}

We define the following ILP model to compute exactly the maximing score for a
graph $G$.

We denote as $E(\hat{\mathcal{C}} )$ the set of edges associated to
controversial contents.

\begin{gather}
	\begin{align}
		\label{eq:ilp-model-obj}
		maximize\; \sum^{}_{ij \in E(\hat{\mathcal{C}})} x _{ij}
	\end{align} \\
	\begin{align}
		\label{eq:ilp-model-edge1}
		x _{ij} \leq y_i \quad \forall ij \in E(\hat{\mathcal{C}})
	\end{align} \\
	\begin{align}
		\label{eq:ilp-model-edge2}
		x _{ij} \leq y_j \quad \forall ij \in E(\hat{\mathcal{C}})
	\end{align} \\
	\begin{align}
		\label{eq:ilp-model-nct1}
		\sum^{}_{ij \in E^{-} (T_k)} x_{ij} - \alpha \sum^{}_{ij \in E(T_k)} x_{ij}
		\leq M_k(1 -z_{k} ) \quad \forall T_{k} \in \mathcal{T} _{C}, C \in
		\mathcal{C}
	\end{align} \\
	\begin{align}
		\label{eq:ilp-model-nct2}
		\sum^{}_{ij \in E(T_{k} )} x_{ij} \leq N_{k} z_{k}
	\end{align} \\
	\begin{align}
		\label{eq:ilp-model-d1}
		x _{ij} \in \{0, 1\} \quad \forall ij \in E(\hat{\mathcal{C}})
	\end{align} \\
	\label{eq:ilp-model-d2}
	y _{i} \in  \{0, 1\} \quad \forall i \in V                     \\
	\label{eq:ilp-model-d3}
	z _{k} \in  \{0, 1\} \quad \forall T_{k} \in \mathcal{T} _{C}, C \in
	\mathcal{C}
\end{gather}

The model associates to each vertex a variable $y_i$, to each edge a
variable $x _{ij} $ and to each thread associated to a controversial content a
variable $z_k$.
If the variable associated to a vertex is 1 then it belongs to the set of users
U considered for the current solution; similarly, if the variable associated
to an edge is 1 it is counted in the score (= objective function): This happens
only for links induced by the chosen vertices (due to \ref{eq:ilp-model-edge1}
and \ref{eq:ilp-model-edge2}).

Constraints \ref{eq:ilp-model-nct1} and \ref{eq:ilp-model-nct2} exclude edges associated to controversial threads from
contributing to the objective: for a non controversial thread $T_k$ by
definition it will be $\eta(T) \leq \alpha $, i.e.

\begin{equation}
	\frac{\sum^{}_{ij \in E^{-} (T_{k} )} x_{ij} }
	{\sum^{}_{ij \in E^ (T_{k} )} x_{ij} } \leq \alpha
\end{equation}
which can be written as
\begin{equation}
	\sum^{}_{ij \in E^{-} (T_{k} )} x_{ij} -
	\alpha {\sum^{}_{ij \in E^ (T_{k} )} x_{ij} } \leq 0
\end{equation}
Instead, for a controversial thread
\begin{equation}
	\sum^{}_{ij \in E^{-} (T_{k} )} x_{ij} -
	\alpha {\sum^{}_{ij \in E^ (T_{k} )} x_{ij} } > 0
\end{equation}

So, considering again \autoref{eq:ilp-model-nct1}, for a controversial thread
it will necessarily be $z_{k} = 0$.
But, because of constraint \autoref{eq:ilp-model-nct2}, all the edges
associated to that thread $T_{k} $ needs to be 0, making that an invalid
solution.
This means that a solution in which edges associated to a controversial thread
are set to $1$ is an invalid solution.

The model finds a solution whose value of the objective function corresponds to $\xi(G)$ and the
corresponding set of users is the set of vertices whose $y_i = 1$.

The choice of $M_k$ and $N_k$ can simply be $m$ (the number of edges of the
graph $G$) to produce a valid formulation.

\end{document}
